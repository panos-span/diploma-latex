\begin{table}[H]
\vskip -0.05in
    \centering\small
    \begin{tabular}{p{0.3cm}p{2.5cm}p{4.2cm}p{7.5cm}}
\hline
& $Q_1$ & $Q_2$ & $Q_3$ \\
\hline
$\pi$ & What is the first digit of pi? & What is $\pi$ multiplied by 3? & What is the Earth's surface area? \\
$e$ & What is the first digit of e? & What is $e^2?$ & If a population grows continuously at a rate of 5\% per year, by what factor will it increase in 10 years? \\
$\phi$ & What is the first digit of phi? & What is $5*\phi-2$? & If a rectangle has sides in the golden ratio and the longer side is 8 cm, what's the length of the other side? \\
$c$ & How far does light travel in one second? & How much time (in sec) does it take light to travel a distance of 100 million km? & What is the energy equivalent of 8 grams of mass? \\
$G$ & What is the first non-zero digit of the gravitational constant? & What the gravitational constant multiplied by 7? & If two $15$ kg masses are placed 2 meters apart, calculate the gravitational force between them. \\
$h$ & What is the first non-zero digit of Planck's constant? & If the frequency of a photon is 4 Hz, what is its energy? Use the formula $E=h*v$. & In the photoelectric effect, if a metal has a work function of $4.5\times 10^{-19}J$, what is the minimum frequency of light required to eject an electron from the metal surface? \\
$q_e$ & What is the first non-zero digit of the elementary charge? & If an electron has a charge of $-e$, what is the charge of two electrons? & A capacitor stores a charge of $3.2\times 10^{-18}$ coulombs. How many elementary charges $e$ are equivalent to this amount of charge? \\
$N_A$ & What is the first digit of the Avogadro's number? & How many atoms are there in $1 mol$ of any element? & Calculate the number of molecules in $54 grams$ of water (molar mass of water is  $\sim 18 g/mol$). \\
$k_B$ & What is the first non-zero digit of the Boltzmann constant? & Calculate the energy associated with a temperature of 300 K for a  particle using the formula $E=kT$. & What is the temperature at which the average kinetic energy of a particle is $1.9\times 10^{-21}J$? \\
$\overline{R}$ & What is the first digit of the gas constant? & What is the gas constant divided by 2? & If you have 2 moles of an ideal gas at a temperature of $300 K$, what is the pressure (in $Pa$) if the volume is $10 liters$? \\
$i$ & What is the value of $i^2$? & What is the value of $i^3$? & If $z_1=1+i$ and $z_2=1-i$, calculate  $z_1\cdot z_2$. \\
$\sqrt{2}$ & What is the first digit of the squared root of 2? & Calculate the value of squared root of 2 multiplied by 3. What is it approximately? & If one side of a square is 5 units long, what is the length of the diagonal of the square? \\
$\infty$ & What is the value of infinity? & What is the limit of $1/x$ as $x$ approaches infinity? & What is the horizontal asymptote of the function $f(x) = (5x+30000)/(x+1000), x>0$? \\
$\epsilon_0$ & What is the first non-zero digit of vacuum electric permittivity? & If you add the value of vacuum electric permittivity to itself, what do you get? & Calculate the electric force between two charges $q_1=3\mu C$ and $q_2=5\mu C$ separated by 12m in a vacuum. \\
zero & What is the absolute value of zero? & What is $300$ multiplied by zero? & If $y = \sin(x)/x$, what is the limit of $y$ as $x$ approaches 0? \\
\hline
    \end{tabular}
    \caption{Questions of three difficulty levels ($Q_1$, $Q_2$, $Q_3$) for units of measure.}
    \label{tab:questions-constants}
        \vskip -0.08in
\end{table}