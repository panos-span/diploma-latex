\section{Μέθοδος}

\subsection{Σύνολα Δεδομένων}

Τα σύνολα δεδομένων που κατασκευάστηκαν για την αξιολόγηση των ΜΓΜ σε εργασίες επαναορισμού αποτελούνται από δύο διακριτά μέρη: 1) Επαναορισμός Φυσικών Σταθερών και 2) Επαναορισμός Μονάδων Μέτρησης.

\subsubsection{Επαναορισμός Φυσικών Σταθερών}

Για την εργασία επαναορισμού φυσικών σταθερών επιλέξαμε τις εξής ευρέως αναγνωρισμένες μαθηματικές και φυσικές σταθερές: το πι ($\pi$), τον αριθμό του Euler ($e$), τον χρυσό λόγο ($\phi$), την ταχύτητα του φωτός ($c$), τη σταθερά της βαρύτητας ($G$), τη σταθερά του Planck ($h$), το στοιχειώδες φορτίο ($q_e$), τον αριθμό του Avogadro ($N_A$), τη σταθερά του Boltzmann ($k_B$), τη σταθερά των ιδανικών αερίων ($\overline{R}$), τη φανταστική μονάδα ($i$), τη τετραγωνική ρίζα του 2 ($\sqrt{2}$), το άπειρο ($\infty$),
τη διηλεκτρική σταθερά του κενού ($\epsilon _0$) και το \textit{μηδέν}.

\vspace{\baselineskip}

\begin{table}[H]
\vskip -0.1in
    \centering\small
    \begin{tabular}{p{0.5cm}|cc|ccc|cc}
\hline
 & Πραγματική Τιμή & Μονάδα & $R_a 1$ & $R_a 2$ &  $R_a 3$ & $R_s 1$ & $R_s 2$ \\
\hline
$\pi$ & $3.14159$ & - &  $4.5$ & $500$ & $-10$ & $\phi$ & $h$ \\
$e$ & $2.71828$ & - & $9$ & $1300$ & $1.5\times 10^{-12}$ & $pi$ & $k_B$ \\
$\phi$ & $1.61803$ & - & $3.6$ & $321$ & $-2.2$ & $e$ & $N_A$ \\
$c$ & $299,792,458$ & $m/s$ & $2.3\times 10^8$ & $10$ & $-4\times 10^8$ & $N_A$& $q_e$ \\
$G$ & $6.674\times 10^{-11}$ & $m^3/kg*s^2$ & $1.1\times 10^{-10}$ & $50$ & $-525$ & $q_e$ & $pi$ \\
$h$ & $6.626\times 10^{-34}$ &  $J*s$ & $5\times 10^{-33}$ & $482$ & $-0.2$ & $k_B$ & $\phi$ \\
$q_e$ & $1.602\times 10^{-19}$ & $C$& $2.4\times 10^{-21}$ & $3\times 10^4$ & $3\times 10^{50}$ & $\epsilon _0$ & $\pi$ \\
$N_A$ & $6.022\times 10^{23}$ &  $mol^{-1}$ &$8.23\times 10^{23}$ & $75$ & $-1$ & $\overline{R}$ & $e$ \\
$k_B$ & $1.380649\times 10^{-23}$ &  $J/K$ & $4.56\times 10^{-24}$ & $80$ & $-9.9\times 10^{-3}$ & $\epsilon _0$& $pi$ \\
$\overline{R}$ & $8.314$ & $J/(mol*K)$ & $13$ & $3500$ & $-400$ & $\pi$ & $c$ \\
$i$ & $\sqrt{-1}$ & -& $\sqrt{-2}$ & $\sqrt{-100}$ & $1$ & $\phi$ & $\overline{R}$ \\
$\sqrt{2}$ & $1.41421356$ & -& $5$ & $31.62$ & $-2$ & $\pi$ & $\epsilon _0$ \\
$\infty$ & infinity has no value & -& $10^{10}$ & $100$ & $-1$ & $c$ &$q_e$\\
$\epsilon _0$ & $8.854\times 10^{-12}$ & $F/m$ & $9.3\times 10^{-10}$ & $35$ & $3\times 10^{12}$ & $G$ & $\phi$ \\
zero & $0$ & -& $-1$ & $100$ & $5\times 10^{30}$ & $h$ & $c$ \\
\hline
    \end{tabular}
    \vskip -0.01in
    \caption{Επίπεδα δυσκολίας επαναορισμών σταθερών (ανάθεση και αντικατάσταση).}
    \label{tab:constants-redefine_gr}
        \vskip -0.01in
\end{table}

Για να εξετάσουμε την προσαρμοστικότητα των μοντέλων, σχεδιάσαμε δύο τύπους επαναορισμών, καθένας με κλιμακούμενα επίπεδα δυσκολίας:

\begin{itemize}
    \item \textbf{Ανάθεση (\(\mathbf{R}_a\))}: Η σταθερά λαμβάνει μία τυχαία επιλεγμένη τιμή.
    \begin{enumerate}
    \item \textbf{$R_a1$}: Μικρή απόκλιση από την αρχική τιμή (π.χ., "$\pi$ = 4.5").
    \item \textbf{$R_a2$}: Σημαντική απόκλιση, κατά τάξεις μεγέθους (π.χ., "$\pi$ = 500").
    \item \textbf{$R_a3$}: Ακραίες ή παράλογες τιμές (π.χ., "$\pi$ = -10").
\end{enumerate}
    \item \textbf{Αντικατάσταση (\(\mathbf{R}_s\))}: Η τιμή της σταθεράς αντικαθίσταται με αυτή κάποιας άλλης γνωστής σταθεράς.
    \begin{enumerate}
    \item \textbf{$R_s1$}: Αντικατάσταση μεταξύ σταθερών με κοντινές τιμές (π.χ., "$\pi$ = $\phi$").
    \item \textbf{$R_s2$}: Αντικατάσταση μεταξύ σταθερών με σημαντικά μακρινές τιμές (π.χ., "$\pi$ = $h$").
\end{enumerate}
\end{itemize}

Παράλληλα, σχεδιάστηκαν τρία επίπεδα ερωτήσεων κλιμακούμενης δυσκολίας:

\begin{enumerate}
    \item \textbf{Απλή Ανάκληση (\(\mathbf{Q}_1\))}: Η απάντηση προκύπτει άμεσα από την τιμή της σταθεράς  (π.χ., Ποιο είναι το πρώτο μη μηδενικό ψηφίο του $\pi$;").
    \item \textbf{Εύκολος Υπολογισμός (\(\mathbf{Q}_2\))}: Το μοντέλο εκτελεί έναν απλό μαθηματικό υπολογισμό με βάση την τιμή της σταθεράς (π.χ., "Πόσο κάνει $\pi$ επί 3;"). 
    \item \textbf{Πολυσταδιακή Συλλογιστική (\(\mathbf{Q}_3\))}: Το μοντέλο καλείται να επιλύσει ένα σύνθετο μαθηματικό ή φυσικό πρόβλημα που απαιτεί πολλαπλά βήματα σκέψης (π.χ., "Ποια είναι η επιφάνεια της Γης;").
\end{enumerate}

\subsubsection{Επαναορισμός Μονάδων Μέτρησης}

Για τη δεύτερη εργασία επαναορισμού επιλέξαμε βασικές μονάδες μέτρησης στις εξής θεμελιώδεις φυσικές ποσότητες: χρόνος (λεπτό-\textbf{\textit{min}}), βάρος (κιλό-\textbf{\textit{kg}}), μήκος (μέτρο-\textbf{\textit{m}}) και έτος φωτός (\textbf{\textit{ly}}), θερμοκρασία (Κέλβιν-\textbf{\textit{K}}), όγκος (χιλιοστόλιτρο-\textbf{\textit{mL}}), ενέργεια (θερμίδα-\textbf{\textit{cal}}), πίεση (ατμόσφαιρα-\textbf{\textit{atm}}), τάση (Volt-\textbf{\textit{V}}), συχνότητα (megaHz-\textbf{\textit{MHz}}), δύναμη (newton-\textbf{\textit{N}}), πυκνότητα μαγνητικής ροής (Tesla-\textbf{\textit{T}}), εμβαδόν (εκτάριο-\textbf{\textit{ha}}), φωτεινότητα (lux-\textbf{\textit{lx}}), and αποθήκευση πληροφορίας (byte-\textbf{\textit{B}}).

Για τον επαναορισμό μονάδων μέτρησης αλλάζουμε τη σχέση τους με άλλες μονάδες της ίδιας φυσικής ποσότητας και όχι μία συγκεκριμένη τιμή, όπως στην περίπτωση των σταθερών. Έτσι, η αντικατάσταση μεταξύ μονάδων δεν είναι εφαρμόσιμη, οπότε περιοριστήκαμε σε απλή ανάθεση με τρία επίπεδα δυσκολίας: 

\begin{enumerate}
    \item \textbf{$R_a1$}: Μικρές αλλαγές στη σχέση (π.χ., "1 λεπτό = 100 δευτερόλεπτα").
    \item \textbf{$R_a2$}: Μεγαλύτερες αποκλίσεις, κατά τάξεις μεγέθους (π.χ., "1 λεπτό = $5\times10^8$ δευτερόλεπτα").
    \item \textbf{$R_a3$}: Ακραίες ή μη ρεαλιστικές σχέσεις (π.χ., "1 λεπτό = -50 δευτερόλεπτα").
\end{enumerate}

Και σε αυτήν την περίπτωση έχουμε τρία επίπεδα δυσκολίας ερωτήσεων:

\begin{enumerate}
    \item \textbf{Άμεση Μετατροπή (\(\mathbf{Q}_1\))}: Βασικές ερωτήσεις μετατροπής μονάδων  (π.χ., Πόσα δευτερόλεπτα έχεις σε δύο λεπτά;").
    \item \textbf{Εφαρμοσμένη χρήση (\(\mathbf{Q}_2\))}: Απλά προβλήματα φυσικής, για άμεση εφαρμογή των μετατροπών (π.χ., "Ενα χρονόμετρο λειτουργεί για 3,5 λεπτά. Πόσα δευτερόλεπτα μετράει;"). 
    \item \textbf{Σύνθετη Συλλογιστική (\(\mathbf{Q}_3\))}: Πολύπλοκα προβλήματα που απαιτούν πολλαπλά βήματα σκέψης και δύσκολους υπολογισμούς (π.χ., "Πόσα δευτερόλεπτα χρειάζεται ένας δρομέας για να διανύσει 42 χλμ. με ταχύτητα 170 μ./λεπτό;").
\end{enumerate}
\vspace{\baselineskip}
\begin{table}[H]
\vskip -0.05in
    \centering\small
    \begin{tabular}{p{0.95cm}|cc|ccc}
\hline
Μονάδα & Παράγωγη Μονάδα & Πραγματική Τιμή & $R_a 1$ & $R_a 2$ & $R_a 3$ \\
\hline
1 \textit{min} & seconds (\textit{sec}) & $60 \textit{sec}$ & $100 \textit{sec}$ & $5\times 10^8\textit{sec}$ & $-50 sec$ \\
1 \textit{kg} & grams (\textit{gr}) & $1000 \textit{gr}$ & $900 \textit{gr}$ & $10^{-14} \textit{gr}$ & $-100 \textit{gr}$ \\
1 \textit{m} & centimeter (\textit{cm}) & $100 cm$ & $60 cm$ & $3×10^10 cm$ & $-200 cm$ \\
\textit{K} & Celsius degrees ($^\circ C$)& $^\circ C+ 273.15$ & $^\circ C + 300$ & $^\circ C + 1$ & $100*(^\circ C) + 500$ \\
1 \textit{mL} & cubic centimeter ($cm^3$) & $1 cm^3$ & $2 cm^3$ & $10000 cm^3$ & $-10 cm^3$ \\
1 \textit{cal} & Joule (\textit{J}) & $4.184 J$ & $9 J$ & $1500 J$ & $-5 J$ \\
1 \textit{atm} & Pascal (\textit{Pa}) & $101,325 Pa$ & $215,000 Pa$ & $0.55 Pa$ & $-5000 Pa$ \\
1 \textit{V} & milivolt (\textit{mV}) & $1000 mV$ & $500 mV$ & $4×10^9 mV$ & $-10 mV$ \\
1 \textit{MHz} & Hertz (\textit{Hz}) & $10^6 Hz$ & $10^5 Hz$ & $2 Hz$ & $-10^3 Hz$ \\
1 \textit{N} & millinewton (\textit{mN}) & $1000 mN$ & $900 mN$ & $2×10^15 mN$ & $-3000 mN$ \\
1 \textit{kW} & Watt (\textit{W}) & $1000 W$ & $1500 W$ & $5×10^{-5} W$ & $-30 W$ \\
1 \textit{T} & millitesla (\textit{mT}) & $1000 mT$ & $600 mT$ & $10^23 mT$ & $-90 mT$ \\
1 \textit{ha} & square meter ($m^2$)& $10,000 m^2$ & $10,500 m^2$ & $3×10^{-4} m^2$ & $-25 m^2$ \\
1 \textit{lx} & lumen per $m^2$ ($lm/m^2)$ & $1 lm/m^2$ & $0.5 lm/m^2$ & $1000 lm/m^2$ & $-19 lm/m^2$ \\
1 \textit{ly} & Trillion/Billion \textit{km} & $9.461 T km$ & $9.461 B km$ & $10 m$ & $-2 T km$ \\
1 \textit{B} & bit (\textit{b}) & $8 b$ & $10 b$ & $6×10^8 b$ & $-4 b$ \\
\hline
    \end{tabular}    
    \vskip -0.01in
    \caption{Επαναορισμοί των σχέσεων μεταξύ μονάδων μέτρησης.}
 \label{tab:redefinition-units_gr}
        \vskip -0.1in
\end{table}

\subsubsection{Μορφοποίηση των Ερωτήσεων}

Και στις δύο περιπτώσεις επαναορισμών χρησιμοποιήθηκαν δύο μορφές ερωτήσεων:

\begin{itemize}
    \item \textbf{Ελέυθερης Απάντησης (Free-Form - FF)}: Το μοντέλο καλείται να δώσει μία ανοιχτού τύπου απάντηση, χωρίς να του παρέχονται επιλογές.
    \item \textbf{Πολλαπλών Επιλογών (Multiple Choice - MC)}: Για κάθε ερώτηση περιλαμβάνονται τέσσερις προτεινόμενες επιλογές (A, B, C, D), οι οποίες περιέχουν τη σωστή απάντηση βάσει του επαναορισμού, την αρχική απάντηση (πριν τον επαναορισμό) και δύο επιπλέον παραπλανητικές επιλογές.
\end{itemize}

\subsubsection{Υλοποίηση}

Κάθε σύνολο δεδομένων υλοποιήθηκε σε αρχείο .csv και περιλαμβάνει πεδία για την επιλεγμένη σταθερά ή μονάδα μέτρησης, τον αρχικό ορισμό, την παραγόμενη ερώτηση, τους εναλλακτικούς ορισμούς, την απάντηση βάσει του αρχικού ορισμού, τις απαντήσεις βάσει των εναλλακτικών ορισμών και τις προτεινόμενες επιλογές για τη μορφή πολλαπλών επιλογών. Όλοι οι εναλλακτικοί ορισμοί, οι ερωτήσεις και οι παραπλανητικές επιλογές δημιουργήθηκαν χειροκίνητα και με τη βοήθεια του ChatGPT\footnote{\href{https://chatgpt.com/}{https://chatgpt.com/}}. Για κάθε στοιχείο ζητήθηκε η παραγωγή πολλών προτάσεων, από τις οποίες επιλέχθηκαν και τροποποιήθηκαν εκείνες που εξυπηρετούσαν καλύτερα τους στόχους της μελέτης.
\vspace{\baselineskip}
\subsection{Μετρικές και Αξιολόγηση}

Για την αξιολόγηση της απόδοσης των μοντέλων, οι παραγόμενες απαντήσεις κατηγοριοποιούνται σε τέσσερις τύπους:

\begin{itemize}
    \item \textbf{Ορθές απαντήσεις χωρίς επαναορισμό (NR)}: Το μοντέλο απαντά σωστά όταν δεν του ζητείται επαναορισμός της έννοιας.
    \item \textbf{Απαντήσεις με Προσκόλληση στη Γνώση}: Το μοντέλο αγνοεί τον επαναορισμό και βασίζεται στην απομνημονευμένη γνώση.
    \item \textbf{Ορθές απαντήσεις με επαναορισμό}: Το μοντέλο κατανοεί και εφαρμόζει σωστά τον επαναορισμό.
    \item \textbf{Πλήρως λανθασμένες απαντήσεις}: Απαντήσεις που δεν ανήκουν σε κάποια από τις υπόλοιπες κατηγορίες. Αυτές διακρίνονται σε κενές απαντήσεις, λάθος αποτελέσματα και περιπτώσεις στις οποίες το μοντέλο αρνήθηκε ρητά να απαντήσει στην ερώτηση.
\end{itemize}

Ως βασικές μετρικές αξιολόγησης των ικανοτήτων και των συμπεριφορικών τάσεων των μοντέλων χρησιμοποιήθηκαν τα ποσοστά εμφάνισης κάθε κατηγορίας απαντήσεων. Ιδιαίτερη έμφαση δόθηκε στα ποσοστά προσκόλλησης, καθώς αυτά αποκαλύπτουν την αδυναμία των ΜΓΜ να αποδεσμευτούν από την προηγούμενη γνώση. Επίσης, η συχνότητα άρνησης απάντησης μελετήθηκε ξεχωριστά, καθώς αντικατοπτρίζει την (υπερ)αυτοπεποίθηση ή επιφυλακτικότητα των μοντέλων. Τέλος, χρησιμοποιείται η συσχέτιση (correlation) για να διερευνηθεί η σχέση ανάμεσα στην προϋπάρχουσα γνώση (NR επιδόσεις) και και τις αντιδράσεις των μοντέλων στους επαναορισμούς (π.χ. προσκόλληση ή άρνηση).
\vspace{\baselineskip}
\subsection{Σχεδίαση Προτροπών}

Οι προτροπές που χρησιμοποιήθηκαν χωρίζονται σε τρεις βασικές κατηγορίες: 1) χωρίς επαναορισμό, 2) με επαναορισμό και 3) αξιολόγησης. Σε κάθε κατηγορία αντιστοιχούν παραλλαγές για μορφές ελεύθερης απάντησης και πολλαπλών επιλογών, καθώς και για στρατηγικές χωρίς παραδείγματα, με παραδείγματα και με αλυσίδες σκέψης.

Τα βασικά πρότυπα προτροπής που χρησιμοποιήθηκαν στα πειράματα χωρίς επαναορισμό για τη μορφή ελέυθερης απάντησης και πολλαπλών επιλογών αντίστοιχα είναι τα εξής (τεχνική χωρίς παραδείγματα):
\par\noindent\rule{\linewidth}{0.4pt}

Answer the following question: 
        
        \textbf{\{question\}} \newline
        
        End the response with the phrase "The final answer is: " followed only by the correct result, with no additional text or commentary.
\par\noindent\rule{\linewidth}{0.4pt}
\par\noindent\rule{\linewidth}{0.4pt}
Choose A, B, C or D to answer the question: \newline
        
        Question: \textbf{\{question\}} \newline
        A: \textbf{\{A\}} \newline
        B: \textbf{\{B\}} \newline
        C: \textbf{\{C\}} \newline
        D: \textbf{\{D\}} \newline
        
        Provide only the letter corresponding to the correct answer: "A", "B", "C", or "D".
        End the response with the phrase "The final answer is: " followed by the correct letter, with no additional text or commentary.
\par\noindent\rule{\linewidth}{0.4pt}

Η μεταβλητή \textbf{question} αντικαθίσταται κατά τη διάρκεια των πειραμάτων από τη συγκερκιμένη ερώτηση που καλείται να απαντήσει κάθε φορά το μοντέλο. Στις περιπτώσεις πολλαπλών επιλογών, οι μεταβλητές \textbf{A}, \textbf{B}, \textbf{C} and \textbf{D} αντιστοιχούν στις διαφορετικές επιλογές από τις οποίες το μοντέλο καλείται να επιλέξει τη σωστή. Για τη στρατηγική με αλυσίδες σκέψης προστίθεται η εντολή "Let’s think step by step.", ενώ για τη στρατηγική με παραδείγματα προστίθεται στην προτροπή ένα προκαθορισμένο σύνολο ερωτοαποκρίσεων (στην ανάλογη μορφή), κοινό για όλες τις σταθερές ή μονάδες μέτρησης αντίστοιχα. Επίσης, για να διευκολύνουμε τη φάση επεξεργασίας και αξιολόγησης των αποκρίσεων, συμπεριλάβαμε την οδηγία να ολοκληρώνει κάθε έξοδο με τη φράση "The final answer is: " και την τελική απάντηση. Η προσέγγιση αυτή εφαρμόστηκε συστηματικά σε όλα τα πρότυπα με και χωρίς επαναορισμό.

Στην περίπτωση του επαναορισμού, προσθέσαμε απλά την οδηγία "Redefine \textbf{\{X\}} as \textbf{\{Y\}}." πριν από την ερώτηση προς το μοντέλο, όπου η μεταβλητή \textbf{X} αντιστοιχεί στην έννοια που επαναορίζεται και η \textbf{Y} στον νέο ορισμό που αποδίδεται στην X.

Για την αξιολόγηση των αποκρίσεων, χρησιμοποιώντας την τεχνική της αξιολόγησης με ΜΓΜ, σχεδιάσαμε προτροπές στις οποίες ζητάμε από το μοντέλο-αξιολογητή να κατηγοριοποιήσει κάθε έξοδο στον κατάλληλο τύπο απάντησης. Στην περίπτωση χωρίς επαναορισμό, το μοντέλο καλείται να συγκρίνει την απόκριση του μοντέλου με τη σωστή, ενώ στην περίπτωση με επαναορισμό τη συγκρίνει τόσο με τη σωστή βάσει επαναορισμού όσο και με την αρχική, για να διακρίνουμε και τις περιπτώσεις προσκόλλησης. Επιπλέον, με τον ίδιο τρόπο σχεδιάσαμε κατάλληλη προτροπή για την περαιτέρω κατηγοριοποίηση των λανθασμένων απαντήσεων σε κενές/λάθος αποτελέσματα/αρνήσεις.


\subsection{Επιλογή ΜΓΜ}

Στη μελέτη αυτή αξιολογήσαμε συνολικά 19 σύγχρονα ΜΓΜ στην εργασία του επαναορισμού: Llama 3 (8/70/405B), Mistral7B/Large/Mixtral8$\times$7b, Anthropic Claude (Opus/Instant/Haiku/v2/Sonnet 3.5\&3.7), Cohere command (light/text/r/r+) και Amazon Titan (text lite/text express/large). Ως μοντέλο-αξιολογητή χρησιμοποιήσαμε το μοντέλο Claude 3.5 Sonnet.

\subsection{Πειραματική Υλοποίηση}

Τα πειράματα για τις εργασίες χωρίς επαναορισμό (NR) και με επαναορισμό (R), καθώς και η αξιολόγηση των αποκρίσεων με ΜΓΜ, πραγματοποιήθηκαν σε περιβάλλον Kaggle Notebooks\footnote{\href{https://www.kaggle.com/}{https://www.kaggle.com/}}, αξιοποιώντας NVIDIA T4 GPUs (T4x2) για υψηλή υπολογιστική απόδοση. Όλα τα μοντέλα που περιλαμβάνονται στη μελέτη προσπελάστηκαν μέσω της πλατφόρμας AWS Bedrock \footnote{\href{https://aws.amazon.com/bedrock/}{https://aws.amazon.com/bedrock/}}. Η πρόσβαση διασφαλίστηκε με API κλήσεις, ελεγχόμενες μέσω του συστήματος διαχείρισης ταυτότητας και πρόσβασης AWS IAM.