\section{Πειραματικά Αποτελέσματα}

\subsection{Επαναορισμός Επιστημονικών Σταθερών}

\subsubsection{Προσκόλληση στις Πραγματικές Τιμές}
Τα αποτελέσματα δείχνουν ότι όλα μοντέλα, ανεξαρτήτως μεγέθους ή αρχιτεκτονικής, παρουσιάζουν σημαντικά ποσοστά προσκόλλησης στις απομνημονευμένες τιμές, ακόμα και όταν τους δίνεται σαφής οδηγία να τις παρακάμψουν. Το φαινόμενο παρατηρείται τόσο στις απαντήσεις ελεύθερης μορφής όσο και σε αυτές των πολλαπλών επιλογών, με το μεγαλύτερο ποσοστό να σημειώνεται από το μοντέλο Llama 405B, φτάνοντας στο 93.33\% των απαντήσεων των πιο δύσκολών επιπέδων ερωτήσεων και αντικατάστασης.

\begin{table}[H]

\centering
\small
\begin{tabular}{p{3cm}|p{0.7cm}p{0.7cm}|p{0.7cm}p{0.7cm}|p{0.7cm}p{0.7cm}|p{0.7cm}p{0.7cm}|p{0.7cm}p{0.7cm}|p{0.7cm}p{0.7cm}}
\hline
\multirow{3}{*}{Μοντέλο} & \multicolumn{6}{c|}{$R_a3$}                                                               & \multicolumn{6}{c}{$R_s2$}                                                                \\ \cline{2-13}
                       & \multicolumn{2}{c}{$Q_1$} & \multicolumn{2}{c}{$Q_2$} & \multicolumn{2}{c|}{$Q_3$} & \multicolumn{2}{c}{$Q_1$} & \multicolumn{2}{c}{$Q_2$} & \multicolumn{2}{c}{$Q_3$} \\ \cline{2-13}
                       & FF           & MC           & FF           & MC           & FF            & MC           & FF           & MC           & FF           & MC           & FF           & MC           \\ \hline

Mistral7B & \textbf{33.33} & \textbf{46.67} & \textbf{33.33} & \textbf{26.67} & 26.67 & 40.0 & 33.33 & 53.33 & 13.33 & 33.33 & 26.67 & 20.0 \\ 
Mixtral8x7B & 33.33 & 33.33 & 26.67 & 26.67 & 20.0 & 33.33 & 26.67 & 46.67 & 40.0 & \textbf{53.33} & 46.67 & \textbf{73.33} \\ 
Mistral Large (123B) & 33.33 & 20.0 & 26.67 & 26.67 & \textbf{53.33} & \textbf{66.67} & \textbf{66.67} & \textbf{53.33} & \textbf{46.67} & 40.0 & \textbf{73.33} & 66.67 \\  \hline
Llama8B & 0.0 & \textbf{26.67} & 0.0 & \textbf{26.67} & 13.33 & 33.33 & 20.0 & 13.33 & \textbf{26.67} & 40.0 & 20.0 & 20.0 \\ 
Llama70B & \textbf{6.67 }& 13.33 & 0.0 & 0.0 & 13.33 & 40.0 & 33.33 & 46.67 & 13.33 & \textbf{46.67} & 33.33 & 73.33 \\ 
Llama405B & 0.0 & 0.0 & 0.0 & 13.33 & \textbf{26.67} & \textbf{53.33} & \textbf{26.67} & \textbf{46.67} & 6.67 & 20.0 & \textbf{53.33} & \textbf{93.33} \\  \hline
Titan lite    & 13.33 & 20.0  & 20.0  & 20.0  & 0.0   & 40.0  & 40.0  & 33.33  & 20.0  & 33.33  & 6.67   & 26.67 \\ 
Titan express & 20.0  & \textbf{26.67} & 13.33  & 13.33  & \textbf{20.0}  & 13.33  & 40.0  & \textbf{53.33}  & \textbf{20.0}  & 20.0  & 33.33  & \textbf{26.67} \\ 
Titan large   & \textbf{26.67}  & 20.0  & \textbf{20.0}  & 6.67  & 13.33  & \textbf{40.0}  & \textbf{60.0}  & 40.0  & 13.33  & \textbf{33.33}  & \textbf{33.33}  & 20.0 \\   \hline
Command r & 0.0 & 6.67 & \textbf{20.0} & \textbf{33.33} & \textbf{26.67} & \textbf{53.33} & \textbf{53.33} & 13.33 & 20.0 & 6.67 & \textbf{33.33} & \textbf{46.67} \\ 
Command r + & 6.67 & 13.33 & 0.0 & 13.33 & 13.33 & 26.67 & 13.33 & 20.0 & 26.67 & 6.67 & 33.33 & 26.67 \\ 
Command light text & 6.67 & 13.33 & 13.33 & 20.0 & 0.0 & 40.0 & 13.33 & 20.0 & \textbf{26.67} & \textbf{20.0} & 13.33 & 13.33 \\ 
Command text & \textbf{13.33} & \textbf{20.0} & 6.67 & 6.67 & 6.67 & 26.67 & 40.0 & \textbf{26.67} & 13.33 & 26.67 & 13.33 & 33.33 \\   \hline
Claude opus & 13.33 & 0.0 & 6.67 & 6.67 & 33.33 & \textbf{46.67} & \textbf{46.67} & \textbf{40.0} & 20.0 & \textbf{26.67} & 53.33 & 73.33 \\ 
Claude instant & 0.0 & 13.33 & 13.33 & \textbf{20.0 } & 26.67 & 46.67 & 33.33 & 20.0 & 33.33 & 40.0 & 46.67 & 60.0 \\ 
Claude haiku & 20.0 & 13.33 & 6.67 & 0.0 & 20.0 & 20.0 & 26.67 & 6.67 & 20.0 & 20.0 & 40.0 & 53.33 \\ 
Claude v2 & 26.67  & 13.33  & \textbf{20.0 } & 0.0 & \textbf{46.67} & 40.0 & 13.33 & 40.0 & \textbf{33.33} & 20.0 & 40.0 & 66.67 \\
Claude 3.5 Sonnet& \textbf{26.67} & \textbf{13.33} & 0.0 & 13.33 & 13.33 & 33.33 & 33.33 & 40.0 & 20.0 & 20.0 & \textbf{60.0 } & \textbf{73.33} \\ 
Claude 3.7 Sonnet & 0.0  & 0.0  & 0.0 & 6.67 & 13.33 & 13.33 & 33.33 & 20.0 & 6.67 & 20.0 & 40.0 & 33.33 \\ 
\hline
\end{tabular}

\caption{Ποσοστά προσκόλλησης όλων των ΜΓΜ με προτροπές χωρίς παραδείγματα για τις πιο δύσκολες περιπτώσεις επαναορισμών σταθερών. Το υψηλότερο ποσοστό για κάθε οικογένεια μοντέλων επισημαίνεται με \textbf{έντονη γραφή}.}
\label{tab:anchored_table_gr}
\end{table}
\begin{table}[h]
\vskip -0.03in
\small
    \centering
    \begin{tabular}{l|p{0.85cm}p{0.85cm}p{0.85cm}|p{0.85cm}p{0.85cm}}
        \hline
        Level & $R_a1$ & $R_a2$ & $R_a3$ & $R_s1$ & $R_s2$ \\ \hline
        \multicolumn{6}{c}{Free-Form (FF)} \\ 
        \hline

        $Q_1$ & \cellcolor{lightdustygreen} -0.458 & -0.071 & 0.008 & 0.199 & -0.016 \\
        $Q_2$ &\cellcolor{lightdustygreen}  -0.502 & \cellcolor{lightdustygreen} -0.573 & \cellcolor{lightdustygreen} -0.472 & 0.107 & 0.019 \\
        $Q_3$ & \cellcolor{lightdustypink} 0.489 & 0.237 & 0.292 & \cellcolor{lightdustypink} 0.666 & \cellcolor{lightdustypink} 0.668 \\ \hline

        \multicolumn{6}{c}{Multiple Choice (MC)} \\ 
        \hline
        $Q_1$ &  \cellcolor{lightdustygreen} -0.642 &  \cellcolor{lightdustygreen} -0.4 &  \cellcolor{lightdustygreen} -0.344 & -0.052 & 0.025 \\ 
        $Q_2$ & -0.275 &  \cellcolor{lightdustygreen} -0.316 & -0.245 & \cellcolor{lightdustypink} 0.41 & 0.151 \\
        $Q_3$ & -0.063 & \cellcolor{lightdustypink} 0.457 & 0.081 & \cellcolor{lightdustypink} 0.666 & \cellcolor{lightdustypink} 0.75 \\ 
        \hline
    \end{tabular}
    \caption{Correlation between average NR correct response rate with anchored response rate for each redefinition and question level in ZS setup. 
    Cells in \textcolor{lightdustypink}{pink} indicate a \textbf{high positive correlation} ($>0.3$), while cells in \textcolor{lightdustygreen}{green} indicate a \textbf{high negative correlation} ($<-0.3$).}
    \label{tab:correlation}
        \vskip -0.12in
\end{table}

Επιπλέον, αναλύουμε τις συσχετίσεις μεταξύ επιδόσεων χωρίς επαναορισμό και ποσοστών προσκόλλησης, οι οποίες αποκαλύτπουν ένα ενδιαφέρον μοτίβο: στις απλούστερες ερωτήσεις οι συσχετίσεις είναι αρνητικές ή πολύ μικρές, υποδηλώνοντας ότι τα μοντέλα που γνωρίζουν καλά τις βασικές τιμές είναι πιο ευέλικτα στην προσαρμογή στους επαναορισμούς σε αυτές τις περιπτώσεις, ενώ στα πιο δύσκολα σενάρια η συσχέτιση γίνεται έντονα θετική, δηλαδή τα μοντέλα που τα πηγαίνουν καλύτερα σε σύνθετες ερωτήσεις υπό κανονικές συνθήκες είναι πιο πιθανό να αποτύχουν να αγνοήσουν τις παγιωμένες γνώσεις τους. Αυτό σημαίνει, ότι τα πιο "έξυπνα" μοντέλα είναι και τα πιο επιρρεπή στο φαινόμενο της προσκόλλησης.





\subsubsection{Αντίστροφη Κλιμάκωση}

Κατά τη δοκιμή μοντέλων διαφορετικών μεγεθών παρατηρήθηκε ένα ενδιαφέρον φαινόμενο, καθώς σε πολλές περιπτώσεις, η αύξηση του μεγέθους των ΜΓΜ οδήγησε σε αυξημένα ποσοστά προσκόλλησης στις προεπιλεγμένες τιμές, και άρα μεγαλύτερη αδυναμία στο να επιλύσουν σωστά προβλήματα επαναορισμών. Μεγαλύτερα μοντέλα, όπως το Mistral Large και το Llama 405B, παρόλο που κατάφεραν καλύτερες επιδόσεις στην πιο "συμβατική" εργασία χωρίς επαναορισμό, εμφάνισαν σημαντικά υψηλότερα ποσοστά προσκόλλησης από τα αντίστοιχα μικρότερά τους, όταν τους ζητήθηκε να υιοθετήσουν εναλλακτικούς ορισμούς σταθερών, ειδικά στα πιο απαιτητικά σενάρια. Το φαινόμενο αυτό επιβεβαιώνεται και σε οπτικά διαγράμματα που απεικονίζουν την αύξηση των ποσοστών προσκόλλησης σε συνάρτηση με το μέγεθος των μοντέλων της ίδιας οικογένειας. Αυτά τα αποτελέσματα επισημαίνουν ότι η αύξηση του αριθμού των παραμέτρων δεν συνεπάγεται απαραίτητα και μεγαλύτερη γνωστική ευελιξία. Αντίθετα, σε αυτήν την εργασία φαίνεται πως ενισχύει την τάση των ΜΓΜ να εμπιστεύονται περισσότερο τη γνώση που έχουν εσωτερικεύσει, ακόμα και όταν αυτό έρχεται σε αντίθεση με τις οδηγίες που τους δίνονται.

\begin{table}[H]
\vskip -0.08in
\centering
\small
\begin{tabular}{p{2.7cm}|p{0.7cm}p{0.6cm}|p{0.7cm}p{0.6cm}|p{0.7cm}p{0.6cm}|p{0.7cm}p{0.6cm}|p{0.7cm}p{0.6cm}|p{0.7cm}p{0.6cm}}
\hline
\multirow{3}{*}{Μοντέλο} & \multicolumn{6}{c|}{$R_a3$}                                                               & \multicolumn{6}{c}{$R_s2$}                                                                \\ \cline{2-13}
                       & \multicolumn{2}{c}{$Q_1$} & \multicolumn{2}{c}{$Q_2$} & \multicolumn{2}{c|}{$Q_3$} & \multicolumn{2}{c}{$Q_1$} & \multicolumn{2}{c}{$Q_2$} & \multicolumn{2}{c}{$Q_3$} \\ \cline{2-13}
                       & NR           & FF           & NR           & FF           & NR            & FF           & NR           & FF           & NR           & FF           & NR           & FF           \\ \hline


Mistral7B & 66.67 & 33.33 & 46.67 & 33.33 & 33.33 & 26.67 & 66.67 & 33.33 & 46.67 & \cellcolor{lightdustypurple}13.33 & 33.33 & \cellcolor{lightdustypurple}26.67 \\ 
Mixtral8x7B & 100.0 & 33.33 & 66.67 & 26.67 & 66.67 & 20.0 & 100.0 & 26.67 & 66.67 & \cellcolor{lightdustypurple}40.0 & 66.67 & \cellcolor{lightdustypurple}46.67 \\ 
Mistral Large (123B) & 93.33 & 33.33 & 73.33 & 26.67 & 53.33 & 53.33 & 93.33 & 66.67 & 73.33 & \cellcolor{lightdustypurple}46.67 & 53.33 & \cellcolor{lightdustypurple}73.33 \\  \hline
Llama8B & 80.0 & 0.0 & 80.0 & 0.0 &  53.33 & \cellcolor{lightdustypurple}13.33 & 80.0 & 20.0 & 80.0 & 26.67 & 53.33 & \cellcolor{lightdustypurple}20.0 \\ 
Llama70B & 93.33 & 6.67 & 80.0 & 0.0 & 80.0 & \cellcolor{lightdustypurple}13.33 & 93.33 & 33.33 & 80.0 & 13.33 & 80.0 & \cellcolor{lightdustypurple}33.33 \\ 
Llama405B & 93.33 & 0.0 & 86.67 & 0.0 & 73.33 & \cellcolor{lightdustypurple}26.67 & 93.33 & 26.67 & 86.67 & 6.67 & 73.33 & \cellcolor{lightdustypurple}53.33 \\  \hline
\end{tabular}
\vskip -0.01in
\caption{Ποσοστό των ορθών απαντήσεων χωρίς επαναορισμό (NR) και ποσοστό προσκόλλησης για ερωτήσεις ανοιχτού τύπου επαναορισμού μονάδων μέτρησης (χωρίς παραδείγματα). Τα χρωματισμένα κελιά υποδεικνύουν ποσοστά προσκόλλησης που αυξάνονται με το μέγεθος των ΜΓΜ.}
\label{tab:anchored_table_NR_gr}
\end{table}

\begin{figure}[H]
\vskip -0.01in
\centering 
\includegraphics[width=0.7\linewidth]{images/model_size_vs_anchored_responses.png}
\caption{Ποσοστά προσκόλλησης για μοντέλα διαφορετικού μεγέθους στην οικογένεια Llama (μορφή πολλαπλών επιλογών).} \label{fig:model_size_vs_anchored_responses_gr} 
\vskip -0.09in
\end{figure}

\begin{figure}[H]
\vskip -0.01in
\centering 
\includegraphics[width=0.7\linewidth]{images/model_size_vs_anchored_responses_mistral.png}
\caption{Ποσοστά προσκόλλησης για μοντέλα διαφορετικού μεγέθους στην οικογένεια Mistral (μορφή πολλαπλών επιλογών).} \label{fig:model_size_vs_anchored_responses_mistral_gr} 
\vskip -0.09in
\end{figure}


\subsubsection{Μορφοποίηση Απαντήσεων}

Η μορφοποίηση των απαντήσεων φαίνεται να επηρεάζει σημαντικά τη συμπεριφορά των μοντέλων, με την περίπτωση των πολλαπλών επιλογών να οδηγεί συστηματικά σε υψηλότερα ποσοστά προσκόλλησης. Αυτό μπορεί πιθανώς να εξηγηθεί από την ίδια τη φύση των δύο μορφοποιήσεων, καθώς, ενώ στις απαντήσεις ανοιχτού τύπου τα μοντέλα καλούνται να σκεφτούν πιο ανεξάρτητα, οι προτεινόμενες σωστές πριν τον επαναορισμό απαντήσεις στις πολλαπλές επιλογές λειτουργούν ως ισχυροί παραπλανητικοί πόλοι. Με άλλα λόγια, όταν τα μοντέλα "βλέπουν" την αρχική σωστή απάντηση μέσα στην προτροπή είναι πιο πιθανό και να την επιλέξουν.

\begin{figure}[H]
\centering
\begin{subfigure}{0.6\textwidth}
        \centering
         \includegraphics[width=0.86\linewidth]{images/mistral-anchored.png}
         \vskip -0.01in
        \caption{Ανάλυση απαντήσεων των μοντέλων Mistral.}
        \label{fig:mistral_gr}
    \end{subfigure}
    \vskip 0.2in
    \begin{subfigure}{0.6\textwidth}
        \centering
        \includegraphics[width=0.86\linewidth]{images/llama-anchored.png}
        \vskip -0.01in
        \caption{Ανάλυση απαντήσεων των μοντέλων Llama.}
        \label{fig:llama_gr}
    \end{subfigure}
    \vskip -0.01in
    \caption{Αποτελέσματα για τα μοντέλα των οικογενειών Mistral και Llama στις ερωτήσεις του τρίτου επιπέδου δυσκολίας με προτροπές χωρίς παραδείγματα. Η σειρά των ράβδων ανά τύπο/επίπεδο επαναορισμού αντιστοιχεί σε αύξουσα σειρά μεγέθους μοντέλου.}
    \label{fig:size_comparison_gr}
\end{figure}


\subsubsection{Τύπος Επαναορισμού}

Εκτός από τις μορφές των απαντήσεων, καθοριστική διαφορά παρατηρείται μεταξύ των δύο τύπων επαναορισμού. Τα πειραματικά αποτελέσματα δείχνουν ότι η αντικατάσταση μεταξύ σταθερών οδηγεί σε αρκετά υψηλότερα ποσοστά προσκόλλησης από την απλή ανάθεση. Υποθέτουμε ότι αυτό συμβαίνει επειδή το σενάριο της αντικατάστασης ενεργοποιεί ισχυρές μνημονικές συνδέσεις για τις δύο γνωστές οντότητες, το οποίο οδηγεί σε σύγχυση και μεγαλύτερη γνωσιακή επιβάρυνση.

\begin{figure}[H]
\vskip -0.01in
\centering 
\includegraphics[width=0.6\linewidth]{images/llama1.png}
\caption{Ανάλυση απαντήσεων για το Llama 70B στο πρώτο επίπεδο ερωτήσεων και όλες τις στρατηγικές προτροπών. Σε κάθε τύπο/επίπεδο επαναορισμού, οι ράβδοι αντιστοιχούν με τη σειρά σε: χωρίς παεαδείγματα, με αλυσίδες σκέψης, με παραδείγματα.} \label{fig:llama1_gr} 
\vskip -0.09in
\end{figure}

\begin{figure}[H]
\vskip -0.01in
\centering 
\includegraphics[width=0.6\linewidth]{images/claude3.png}
\caption{Ανάλυση απαντήσεων για το Claude 3.5 Sonnet στο τρίτο επίπεδο ερωτήσεων και όλες τις στρατηγικές προτροπών. Σε κάθε τύπο/επίπεδο επαναορισμού, οι ράβδοι αντιστοιχούν με τη σειρά σε: χωρίς παεαδείγματα, με αλυσίδες σκέψης, με παραδείγματα.} \label{fig:claude3_gr} 
\vskip -0.09in
\end{figure}

\subsubsection{Λειτουργία Εκτεταμένης Σκέψης}

Το μοντέλο Claude 3.7 Sonnet της Anthropic διαθέτει την επιπλέον λειτουργία εκτεταμένης σκέψης (extended thinking), η οποία επιτρέπει στο μοντέλο να αναλύει τα προβλήματα πιο διαξοδικά, παράγοντας μπλοκ σκέψης που αποτυπώνουν την εσωτερική του συλλογιστική πορεία. Δοκιμάσαμε αυτήν τη λειτουργία επαναλαμβάνοντας τα ίδια πειράματα και συγκρίναμε τα αποτελέσματα με αυτά της βασικής (standard) περίπτωσης. Βρήκαμε πως, αν και η λειτουργία εκτεταμένης σκέψης μειώνει ελαφρώς τα ποσοστά προσκόλλησης σε ορισμένες περιπτώσεις, η συνολική της επίδραση είναι αμελητέα. Αυτό δείχνει ότι ακόμα και με ενισχυμένες δυνατότητες συλλογιστικής, το μοντέλο εξακολουθεί να δυσκολεύεται να ανταποκριθεί σε εννοιολογικά απαιτητικές οδηγίες επαναορισμού, αποκαλύπτοντας περιορισμούς στη γνωστική του ευελιξία.


\begin{figure}[H]
\begin{subfigure}{\textwidth}
        \centering
         \includegraphics[width=0.86\linewidth]{images/no_thinking_constants_FF.png}
         \vskip -0.01in
        \caption{Claude 3.7 Sonnet χωρίς Thinking σε ερωτήσεις ανοιχτού τύπου.}
        \label{fig:no_thinking_constants_FF_gr}
    \end{subfigure}
    
    \begin{subfigure}{\textwidth}
        \centering
        \includegraphics[width=0.86\linewidth]{images/no_thinking_constants_MC.png}
        \vskip -0.01in
        \caption{Claude 3.7 Sonnet χωρίς Thinking σε ερωτήσεις πολλαπλών επιλογών.}
        \label{fig:no_thinking_constants_MC_gr}
    \end{subfigure}
        \begin{subfigure}{\textwidth}
        \centering
        \includegraphics[width=0.86\linewidth]{images/thinking_constants_FF.png}
        \vskip -0.01in
        \caption{Claude 3.7 Sonnet με Thinking σε ερωτήσεις ανοιχτού τύπου.}
        \label{fig:thinking_constants_FF_gr}
    \end{subfigure}

     \begin{subfigure}{\textwidth}
        \centering
        \includegraphics[width=0.86\linewidth]{images/thinking_constants_MC.png}
        \vskip -0.01in
        \caption{Claude 3.7 Sonnet με Thinking σε ερωτήσεις πολλαπλών επιλογών.}
        \label{fig:thinking_constants_MC_gr}
        \end{subfigure}
    \caption{Ανάλυση απαντήσεων Claude 3.7 Sonnet χωρίς και με Thinking.}
    \label{fig:thinking_gr}
\end{figure}


\subsubsection{Επίδραση Προτροπών}



\begin{figure} [H]
\vskip -0.08in
\centering 
\includegraphics[width=0.7\linewidth]{images/models_zs_fs_cot_v4.png} 
\vskip -0.07in
\caption{Σύγκριση των ποσοστών προσκόλλησης για τις ερωτήσεις $Q_3$ και το επίπεδο επαναορισμών $R_{s2}$ για όλα τα ΜΓΜ.} \label{fig:model_zs_fs_cot_gr} 
\vskip -0.1in
\end{figure}

Η μελέτη της επίδρασης διαφορετικών τεχνικών προτροπής (χωρίς παραδείγματα, με παραδείγματα, με αλυσίδες σκέψης) έδειξε πως η συμπεριφορά των ΜΓΜ επηρεάζεται, αλλά όχι με συνεπή ή καθοριστικό τρόπο. Ενδιαφέρον παρουσιάζει το γεγονός ότι η τεχνική αλυσίδων σκέψης δεν μειώνει συστηματικά τα ποσοστά προσκόλλησης των μοντέλων, παρόλο που γενικά είναι γνωστό ότι ενισχύει τη συλλογιστική ικανότητα των μοντέλων μέσω της βηματικής επίλυσης \cite{kojima2023largelanguagemodelszeroshot}. Αντίθετα, η προτροπή με παραδείγματα είναι πιο αποτελεσματική στην περίπτωσή μας, αφού πάνω από τα μισά μοντέλα εμφανίζουν καλύτερη επίδοση σε αυτες τις συνθήκες, πιθανώς επειδή οι επιδείξεις σωστών απαντήσεων με ενσωματωμένους επαναορισμούς προσφέρουν στα ΜΓΜ ένα ισχυρό πλαίσιο αναφοράς προς μίμηση. Ωστόσο, λόγω της μεγάλης διακύμανσης των αποτελεσμάτων, συμπεραίνουμε πως η προσκόλληση είναι ένα φαινόμενο σχετικά ανεπηρέαστο από τις παρεμβάσεις μέσω διαφορετικών τεχνικών προτροπής.

\subsubsection{Άρνηση Απόκρισης}



\begin{table}[H]
\centering
\small
\begin{tabular}{p{2.4cm}|c|c|c}
\hline
Μοντέλο                  & Προτροπή & FF & MC \\ \hline

      \multirow{3}{*}{Mistral7B}  &  ZS & \underline{6.57 ± 11.99} & 13.34 ± 18.07 \\
 &  CoT & 5.63 ± 8.89 & \underline{15.62 ± 16.45} \\
 &  FS & \textbf{3.7 ± 7.58} & \textbf{10.07 ± 15.25} \\
\hline
\multirow{3}{*}{Mixtral8x7B}  &  ZS & \underline{18.0 ± 22.8} & 8.61 ± 16.97 \\
 &  CoT & 9.22 ± 16.82 & \underline{15.5 ± 17.63} \\
 &  FS & \textbf{10.98 ± 17.03} & \textbf{5.95 ± 18.79} \\
\hline
\multirow{3}{*}{Mistral Large}  &  ZS & \underline{16.33 ± 33.69} & 1.67 ± 6.24 \\
 &  CoT & \textbf{8.33 ± 18.51} & \textbf{0 ± 0} \\
 &  FS & 14.35 ± 26.96 & \underline{1.33 ± 4.99} \\
\hline
\multirow{3}{*}{Llama8B}  &  ZS & \underline{55.54 ± 24.37} & \underline{40.05 ± 18.58} \\
 &  CoT & 35.25 ± 23.33 & 32.89 ± 23.21 \\
 &  FS & \textbf{2.41 ± 6.64} & \textbf{0 ± 0}\\
\hline
\multirow{3}{*}{Llama70B}  &  ZS & \underline{38.66 ± 29.92} & 5.56 ± 14.49 \\
 &  CoT & 9.17 ± 17.36 & \underline{13.33 ± 27.35} \\
 &  FS & \textbf{0 ± 0} & \textbf{0 ± 0}\\
\hline
\multirow{3}{*}{Llama405B}  &  ZS & \underline{1.33 ± 4.99}  & \textbf{0 ± 0}\\
 &  CoT & \textbf{0 ± 0} & \textbf{0 ± 0} \\
 &  FS & \textbf{0 ± 0} & \textbf{0 ± 0} \\
\hline
\multirow{3}{*}{Titan lite}  &  ZS & \textbf{1.56 ± 3.19} & \textbf{0 ± 0}\\
 &  CoT & \underline{3.03 ± 5.66} & \textbf{0 ± 0}\\
 &  FS & 2.54 ± 5.39 & \textbf{0 ± 0} \\
\hline
\multirow{3}{*}{Titan express}  &  ZS & 0.56 ± 2.08  & \textbf{0 ± 0} \\
 &  CoT & \underline{1.9 ± 7.13} & \textbf{0 ± 0} \\
 &  FS & \textbf{0 ± 0} & \textbf{0 ± 0} \\ \hline
\multirow{3}{*}{Titan large}  &  ZS & \underline{2.0 ± 5.42}  & \textbf{0 ± 0}\\
 &  CoT & \textbf{0 ± 0} & \textbf{0 ± 0}\\
 &  FS & \textbf{0 ± 0} & \textbf{0 ± 0}\\
\hline
\multirow{3}{*}{Command text}  &  ZS & \underline{3.33 ± 9.03}  & \textbf{0 ± 0} \\
 &  CoT & \textbf{0 ± 0} & \textbf{0 ± 0}\\
 &  FS & 0.83 ± 3.12  & \textbf{0 ± 0} \\
\hline
\multirow{3}{*}{Claude instant}  &  ZS & \underline{1.69 ± 4.36}  & \textbf{0 ± 0} \\
 &  CoT & \textbf{0 ± 0} & \textbf{0 ± 0}\\
 &  FS & 4.07 ± 12.58  & \textbf{0 ± 0} \\
\hline
\multirow{3}{*}{\shortstack{Claude v2}}  &  ZS & \underline{20.48 ± 26.25} & 4.83 ± 9.29 \\
 &  CoT & 14.31 ± 24.39 & \underline{10.0 ± 27.08} \\
 &  FS & \textbf{8.91 ± 24.75} & \textbf{3.17 ± 8.81} \\
\hline
\end{tabular}
\caption{Μέσα ποσοστά άρνησης για όλα τα ΜΓΜ (μικρότερες τιμές σε \textbf{έντονη γραφή} και μεγαλύτερες τιμές \underline{υπογραμμισμένες}). Δεν συμπεριλαμβάνονται τα μοντέλα που σημείωσαν μηδενικά ποσοστά σε όλες τις περιπτώσεις.}
\label{tab:refusal_rate_main_gr}
\vskip -0.1in
\end{table}

Παρόλο που το φαινόμενο της προσκόλλησης ήταν το κύριο αντικέιμενο μελέτης αυτής της εργασίας, μια επίσης ενδιαφέρουσα συμπεριφορά παρατηρήθηκε σε πολλές περιπτώσεις, όταν τα μοντέλα αρνούνταν ρητά να απαντήσουν σε ερωτήσεις που σχετίζονται με τον επαναορισμό γνωστών εννοιών, κρίνοντάς τες μη έγκυρες, παράλογες ή παραπλανητικές. Το φαινόμενο αυτό ήταν πιο έντονο σε συγκεκριμένες οικογένειες ΜΓΜ, όπως στις Mistral και Llama, και ειδικά στις εκδόσεις τους με τον μικρότερο αριθμό παραμέτρων. Αντίθετα, μοντέλα απο τις οικογένειες Claude, Titan και Cohere παρουσιάζουν σημαντικά μικρότερα--και συχνά μηδενικά-- ποσοστά τέτοιων αποκρίσεων. Αναφορικά με τις τεχνικές προτροπής, η προσέγγιση με παραδείγματα φαίνεται να μειώνει πιο αισθητά τα ποσοστά άρνησης, το οποίο είναι αναμενόμενο αφού μέσα από τα παραδείγματα κανονικοποιείται η διαδικασία του επαναορισμού. Επίσης, μετρώντας τις συσχετίσεις μεταξύ ακρίβειας χωρίς επαναορισμό και άρνησης (0.144 για ελεύθερη απάντηση και 0.039 για πολλαπλές επιλογές κατά μέσο όρο), συμπεραίνουμε πως αυτή η συμπεριφορά δεν σχετίζεται ισχυρά με τις βασικές ικανότητες λογικής των ΜΓΜ, άλλα μάλλον περοσσότερο με την κλίμακά τους. Τα μεγαλύτερα μοντέλα τείνουν να αρνούνται να απαντήσουν λιγότερο συχνά, επιδεικνύοντας μιά μορφή αυξημένης αυτοπεποίθησης που τα ωθεί να προσπαθούν, ακόμα και αν τελικά αποτυγχάνουν.

\subsection{Επαναορισμός Μονάδων Μέτρησης}

\subsubsection{Προσκόλληση στις Πραγματικές Τιμές}

Το φαινόμενο της προσκόλλησης στους πραγματικούς ορισμούς παραμένει και στην περίπτωση των επαναορισμών μονάδων μέτρησης. Όλα τα μοντέλα, σε διαφορετικό βαθμό, παράγουν απαντήσεις που βαζίζονται στις προϋπάρχουσες γνώσεις τους, αγνοώντας την οδηγία επαναορισμού. Τα ποσοστά προσκόλλησης, βέβαια, είναι γενικά χαμηλότερα από τα αντίστοιχα των σταθερών, με κάποια μοντέλα (κυρίως από τις οικογένειες Command και Claude) να πετυχαίνουν ακόμα και μηδενικά αποτελέσματα σε πιο εύκολα σενάρια ερωτήσεων ανοιχτού τύπου.

\begin{table}[H]
\centering
\small
\begin{tabular}{l|p{0.7cm}p{0.6cm}|p{0.7cm}p{0.6cm}|p{0.7cm}p{0.6cm}|p{0.7cm}p{0.6cm}|p{0.7cm}p{0.6cm}|p{0.7cm}p{0.6cm}}
\hline
\multirow{3}{*}{Μοντέλο} & \multicolumn{6}{c|}{$R_a2$}                                                               & \multicolumn{6}{c}{$R_a3$}                                                                \\ \cline{2-13}
                       & \multicolumn{2}{c}{$Q_1$} & \multicolumn{2}{c}{$Q_2$} & \multicolumn{2}{c|}{$Q_3$} & \multicolumn{2}{c}{$Q_1$} & \multicolumn{2}{c}{$Q_2$} & \multicolumn{2}{c}{$Q_3$} \\ \cline{2-13}
                       & FF           & MC           & FF           & MC           & FF            & MC           & FF           & MC           & FF           & MC           & FF           & MC           \\ \hline

Mistral7B & 0.0 & 37.5 & 25.0 & 25.0 & 18.75 & 56.25 & \textbf{62.5} & 25.0 & \textbf{31.25} & \textbf{37.5} & 31.25 & 25.0 \\ 
Mixtral8x7B & \textbf{6.25} & 31.25 & \textbf{31.25} & \textbf{37.5} & 31.25 & 37.5 & 6.25 & \textbf{31.25} & 6.25 & 31.25 & \textbf{31.25}  & \textbf{50.0 } \\ 
Mistral Large & 0.0 & \textbf{37.5} & 6.25 & 37.5 & 12.5 & 56.25 & 0.0 & 25.0 & 12.5 & \textbf{37.5} & 12.5 & 43.75 \\ \hline
Llama8B & 0.0 & \textbf{25.0} & \textbf{6.25} & \textbf{31.25} & 12.5 & 31.25 & \textbf{6.25} & \textbf{31.25}  & \textbf{12.5} & \textbf{50.0} & \textbf{25.0} & 50.0 \\ 
Llama70B & 0.0 & 6.25 & \textbf{6.25} & \textbf{31.25} & \textbf{25.0} & \textbf{56.25} & 0.0 & 18.75 & 0.0 & \textbf{50.0} & 12.5 & \textbf{62.5} \\ 
Llama405B & 0.0 & 0.0 & 0.0 & \textbf{31.25} & 12.5 & 37.5 & 0.0 & 0.0 & 6.25 & 25.0 & \textbf{25.0} & 31.25 \\ \hline
Titan lite & 6.25 & \textbf{25.0} & 12.5 & \textbf{31.25} & 12.5 & \textbf{25.0} & 25.0 & \textbf{31.25} & 25.0 & 12.5 & 0.0 & 18.75 \\
Titan express & 18.75 & \textbf{25.0} & \textbf{25.0} & 18.75 & 12.5 & \textbf{25.0} & \textbf{43.75} & 25.0 & 31.25 & 12.5 & 6.25 & 18.75 \\
Titan large & \textbf{31.25} & 12.5 & 12.5 & \textbf{31.25} & \textbf{18.75} & \textbf{25.0} & 25.0 & 12.5 & \textbf{37.5} & \textbf{31.25} & 6.25 & \textbf{25.0} \\ \hline
Command r & \textbf{12.5 }& 18.75 & \textbf{12.5} & \textbf{31.25}  & 25.0 & 18.75 & 6.25 & 25.0 & \textbf{12.5} & 18.75 & \textbf{12.5} & 31.25 \\ 
Command r+ & 6.25 & \textbf{43.75} & 0.0 & 25.0 & \textbf{37.5} & \textbf{50.0} & \textbf{6.25} & \textbf{31.25}  & 0.0 & \textbf{31.25} & 0.0 & 25.0 \\ 
Command light text & 6.25 & 12.5 & 0.0 & 25.0 & 6.25 & 25.0 & 12.5 & 25.0 & 6.25 & 31.25 & 0.0 & \textbf{50.0} \\ 
Command text & 12.5 & 12.5 & 12.5 & 18.75 & 0.0 & 18.75 & 0.0 & 31.25 & 12.5 & 12.5 & 0.0 & 43.75 \\ \hline
Claude opus & 0.0 & 0.0 & 0.0 & 6.25 & 12.5 & 25.0 & 0.0 & 0.0 & 0.0 & 0.0 & 0.0 & 6.25 \\ 
Claude instant & \textbf{6.25} & \textbf{25.0} & \textbf{12.5} & 25.0 & 0.0 & \textbf{43.75} & 0.0 & \textbf{43.75} & 0.0 & \textbf{37.5} & 6.25 & \textbf{31.25} \\ 
Claude haiku & 0.0 & 18.75 & 0.0 & 12.5 & 6.25 & 31.25 & 0.0 & 6.25 & 0.0 & 6.25 & \textbf{18.75} & \textbf{31.25} \\ 
Claude v2 & \textbf{6.25} & 18.75 & 6.25 & \textbf{31.25} & \textbf{18.75} & 31.25 & \textbf{6.25} & 0.0 & \textbf{6.25} & 25.0 & 6.25 & 12.5 \\
Claude 3.5 Sonnet & 0.0 & 0.0 & 0.0 & 12.5 & 6.25 & 6.25 & 0.0 & 0.0 & 0.0 & 6.25 & 0.0 & 0.0 \\ 
Claude 3.7 Sonnet & 0.0 & 0.0 & 0.0 & 0.0 & 0.0 & 0.0 & 0.0 & 0.0 & 0.0 & 0.0 & 0.0 & 0.0 \\
\hline
\end{tabular}

\caption{Ποσοστά προσκόλλησης όλων των ΜΓΜ με πρτοτροπές χωρίς παραδείγματα για τις πιο δύσκολες περιπτώσεις επαναορισμών μονάδων μέτρησης. Το υψηλότερο ποσοστό για κάθε οικογένεια μοντέλων επισημαίνεται με \textbf{έντονη γραφή}.}
\label{tab:anchored_table_units_gr}
\end{table}




\subsubsection{Αντίστροφη Κλιμάκωση}


\begin{table}[H]
\centering
\small
\begin{tabular}{l|p{0.7cm}p{0.6cm}|p{0.7cm}p{0.6cm}|p{0.7cm}p{0.6cm}|p{0.7cm}p{0.6cm}|p{0.7cm}c|p{0.7cm}p{0.6cm}}
\hline
\multirow{3}{*}{Model} & \multicolumn{6}{c|}{$R_a2$}                                                               & \multicolumn{6}{c}{$R_a3$}                                                                \\ \cline{2-13}
                       & \multicolumn{2}{c}{$Q_1$} & \multicolumn{2}{c}{$Q_2$} & \multicolumn{2}{c|}{$Q_3$} & \multicolumn{2}{c}{$Q_1$} & \multicolumn{2}{c}{$Q_2$} & \multicolumn{2}{c}{$Q_3$} \\ \cline{2-13}
                       & NR           & FF           & NR           & FF           & NR            & FF           & NR           & FF           & NR           & FF           & NR           & FF           \\ \hline


Mistral 7B & 81.25 & 0.0 & 56.25 & 25.0 & 43.75 & 18.75 & 81.25 & 62.5 & 56.25 & 31.25 & 43.75 & 31.25 \\ 
Mixtral8x7B & 87.5 & 6.25 & 81.25 & 31.25 & 62.5 & 31.25 & 87.5 & 6.25 & 81.25 & 6.25 & 62.5 & 31.25 \\ 
Mistral Large & 93.75 & 0.0 & 93.75 & 6.25 & 81.25 & 12.5 & 93.75 & 0.0 & 93.75 & 12.5 & 81.25 & 12.5 \\ \hline
Llama8B & 75.0 & 0.0 & 56.25 & 6.25 & 6.25 & 12.5 & 75.0 & 6.25 & 56.25 & 12.5 & 6.25 & 25.0 \\ 
Llama70B & 100.0 & 0.0 & 81.25 & 6.25 & 56.25 & 25.0 & 100.0 & 0.0 & 81.25 & 0.0 & 56.25 & 12.5 \\ 
Llama405B & 100.0 & 0.0 & 93.75 & 0.0 & 56.25 & 12.5 & 100.0 & 0.0 & 93.75 & 6.25 & 56.25 & 25.0 \\ \hline
Titan lite & 37.5 & 6.25 & 18.75 & 12.5 & 6.25 & 12.5 & 37.5 & 25.0 & 18.75 & 25.0 & 6.25 & 0.0 \\
Titan express & 75.0 & 18.75 & 37.5 & 25.0 & 6.25 & 12.5 & 75.0 & 43.75 & 37.5 & 31.25 & 6.25 & 6.25 \\
Titan large & 68.75 & 31.25 & 68.75 & 12.5 & 25.0 & 18.75 & 68.75 & 25.0 & 68.75 & 37.5 & 25.0 & 6.25 \\ \hline
Command r & 75.0 & 12.5 & 56.25 & 12.5 & 18.75 & 25.0 & 75.0 & 6.25 & 56.25 & 12.5 & 18.75 & 12.5 \\ 
Command r+ & 87.5 & 6.25 & 93.75 & 0.0 & 81.25 & 37.5 & 87.5 & 6.25 & 93.75 & 0.0 & 81.25 & 0.0 \\ 
Command light text & 31.25 & 6.25 & 6.25 & 0.0 & 0.0 & 6.25 & 31.25 & 12.5 & 6.25 & 6.25 & 0.0 & 0.0 \\ 
Command text & 62.5 & 12.5 & 50.0 & 12.5 & 25.0 & 0.0 & 62.5 & 0.0 & 50.0 & 12.5 & 25.0 & 0.0 \\ \hline
Claude opus & 100.0 & 0.0 & 75.0 & 0.0 & 56.25 & 12.5 & 100.0 & 0.0 & 75.0 & 0.0 & 56.25 & 0.0 \\
Claude instant & 75.0 & 6.25 & 81.25 & 12.5 & 43.75 & 0.0 & 75.0 & 0.0 & 81.25 & 0.0 & 43.75 & 6.25 \\ 
Claude haiku & 100.0 & 0.0 & 93.75 & 0.0 & 81.25 & 6.25 & 100.0 & 0.0 & 93.75 & 0.0 & 81.25 & 18.75 \\ 
Claude v2 & 93.75 & 6.25 & 68.75 & 6.25 & 25.0 & 18.75 & 93.75 & 6.25 & 68.75 & 6.25 & 25.0 & 6.25 \\ 
Claude 3.5 Sonnet & 100.0 & 0.0 & 87.5 & 0.0 & 87.5 & 6.25 & 100.0 & 0.0 & 87.5 & 0.0 & 87.5 & 0.0 \\ 
Claude 3.7 Sonnet & 100.0 & 0.0 & 87.5 & 0.0 & 93.75 & 0.0 & 100.0 & 0.0 & 87.5 & 0.0 & 93.75 & 0.0 \\ 
\hline
\end{tabular}

\caption{The percentage of correct responses with no redefinition (NR) and the anchored response rate for units of measure redefinitions regarding free-form (FF) responses using ZS prompting.}
\label{tab:anchored_table_NR_units}
\end{table}




Και στην εργασία επαναορισμού μονάδων μέτρησης εμφανίζονται τάσεις αντίστροφης κλιμάκωσης. Σε αρκετές περιπτώσεις, μεγαλύτερα μοντέλα (όπως Mistral Large, Titan Large και Llama 405B) εμφανίζουν αυξημένα ποσοστά προσκόλλησης σε σχέση με μικρότερα εντός των ίδιων οικογενειών, παρόλο που πετυχαίνουν καλύτερες επιδόσεις στις αντίστοιχες εργασίες χωρίς επαναορισμούς. Αν και το φαινόμενο δεν είναι τόσο έντονο όσο στην περίπτωση των φυσικών σταθερών, παραμένει αξιοσημείωτο, καθώς, για άλλη μια φορά, η αυξημένη λογική ικανότητα των μεγαλύτερων μοντέλων περιέργως δεν μεταφράζεται και σε καλύτερη προσαρμογή σε επαναορισμένες συνθήκες.

\begin{figure}[H]
\begin{subfigure}{\textwidth}
        \centering
         \includegraphics[width=0.86\linewidth]{images/mistral7b_stacked_bars-units.png}
        \caption{Ανάλυση απαντήσεων του  Mistral7B πριν και μετά τους επαναορισμούς μονάδων μέτρησης.}
        \label{fig:mistral7b_MC-units_gr}
    \end{subfigure}
    
    \begin{subfigure}{\textwidth}
        \centering
        \includegraphics[width=0.86\linewidth]{images/mistral_large_stacked_bars-units.png}
        \caption{Ανάλυση απαντήσεων του  Mistral Large πριν και μετά τους επαναορισμούς μονάδων μέτρησης.}
        \label{fig:mistral_large_MC-units_gr}
    \end{subfigure}
    \caption{Σύγκριση των απαντήσεων των Mistral7B και Mistral Large (123B)  σε ερωτήσειε πολλαπλών επιλογών για επαναορισμούς μονάδων μέτρησης.}
    \label{fig:mistral_all-units_gr}
\end{figure}

\subsubsection{Μορφοποίηση Απαντήσεων}

Για άλλη μια φορά, η μορφή πολλαπλών επιλογών ενισχύει σημαντικά το φαινόμενο της προσκόλλησης στην απομνημονευμένη γνώση σε σχέση με την ελεύθερη απάντηση, με ποσοστά να ανεβαίνουν, για παράδειγμα, από 12.5\% στα 62.5\%. Το γεγονός αυτό οφείλεται στην έκθεση του μοντέλου στην καθιερωμένη σχέση μεταξύ των μονάδων μέτρησης μέσα από τις προτεινόμενες επιλογές, η οποία ενισχύει τη σύγκρουση μεταξύ της οδηγίας και της προϋπάρχουσας από την εκπαίδευσή του γνώσης.

\begin{figure}[H]
\centering
\begin{subfigure}{0.6\textwidth}
        \centering
         \includegraphics[width=0.86\linewidth]{images/mistral-anchored-units.png}
         \vskip -0.01in
        \caption{Ανάλυση απαντήσεων των μοντέλων Mistral.}
        \label{fig:mistral_gr}
    \end{subfigure}
    \vskip 0.2in
    \begin{subfigure}{0.6\textwidth}
        \centering
        \includegraphics[width=0.86\linewidth]{images/llama-anchored-units.png}
        \vskip -0.01in
        \caption{Ανάλυση απαντήσεων των μοντέλων Llama.}
        \label{fig:llama_gr}
    \end{subfigure}
    \vskip -0.01in
    \caption{Αποτελέσματα για τα μοντέλα των οικογενειών Mistral και Llama στις ερωτήσεις του τρίτου επιπέδου δυσκολίας με προτροπές χωρίς παραδείγματα. Η σειρά των ράβδων ανά τύπο/επίπεδο επαναορισμού αντιστοιχεί σε αύξουσα σειρά μεγέθους μοντέλου.}
    \label{fig:size_comparison_units_gr}
\end{figure}

\subsubsection{Επίδραση Προτροπών}

Αντίθετα με τα αποτελέσματα των σταθερών, στην περίπτωση των μονάδων μέτρησης φαίνεται πως η τεχνική με αλυσίδες σκέψης είναι πιο αποτελεσματική για τον περιορισμό του φαινομένου της προσκόλλησης.

\begin{table}[H]
\small
    \centering
    \begin{tabular}{l|ccc}
        \hline
        Επίπεδο & $R_{a1}$ & $R_{a2}$ & $R_{a3}$ \\ \hline
        & \multicolumn{3}{c}{Ελεύθερης Απάντησης (FF)} \\ 
        \hline
        $Q_1$ & -0.295 & \cellcolor{lightdustygreen} -0.403   & \cellcolor{lightdustygreen} -0.33 \\ 
        $Q_2$ & \cellcolor{lightdustygreen} -0.361 & -0.247 & \cellcolor{lightdustygreen} -0.479 \\ 
        $Q_3$ & -0.063 & 0.19 & 0.14 \\
        
        \hline
        & \multicolumn{3}{c}{Πολλαπλών Επιλογών (MC)} \\ \hline
        $Q_1$ & \cellcolor{lightdustygreen} -0.49 & -0.149  & \cellcolor{lightdustygreen} -0.542 \\ 
        $Q_2$ & -0.159 & -0.023 & 0.08 \\ 
        $Q_3$ & 0.248 & \cellcolor{lightdustypink} 0.338 & -0.127 \\

        \hline
    \end{tabular}
    \caption{Μέση τιμή συσχέτισης μεταξύ επίδοσης στην εργασία χωρίς επαναορισμό και ποσοστών προσκόλλησης για τη στρατηγική χωρίς παραδείγματα. Τα κελιά με  \textcolor{lightdustypink}{ροζ} χρώμα υποδηλώνουν \textbf{υψηλή θετική συσχέτιση} ($>0.3$), ενώ αυτά με \textcolor{lightdustygreen}{πράσινο} χρώμα \textbf{υψηλή αρνητική συσχέτιση} ($<-0.3$).}
    \label{tab:correlation_zs-units_gr}
\end{table}
\begin{table}[H]
\small
    \centering
    \begin{tabular}{l|ccc}
        \hline
        Level & $R_{a1}$ & $R_{a2}$ & $R_{a3}$ \\ \hline
        & \multicolumn{3}{c}{Free-Form (FF)} \\ 
        \hline
        $Q_1$ & \cellcolor{lightdustygreen} -0.32 & \cellcolor{lightdustygreen} -0.442 & -0.161 \\
        $Q_2$ & \cellcolor{lightdustygreen} -0.404 & -0.231 & 0.039 \\
        $Q_3$ & 0.128 & -0.042 & 0.279 \\
        
        \hline
                & \multicolumn{3}{c}{Multiple Choice (MC)} \\ \hline

                
$Q_1$ & \cellcolor{lightdustygreen} -0.332 & 0.058 & \cellcolor{lightdustygreen} -0.593 \\
$Q_2$ & 0.135 & 0.131 & 0.266 \\
$Q_3$ & \cellcolor{lightdustypink} 0.314 & \cellcolor{lightdustypink} 0.49 & 0.101 \\

        \hline
    \end{tabular}
    \caption{Correlation between model performance before redefinition with the percentage of anchored answers for each type of unit of measure redefinition and question level in FS setup. 
    Cells highlighted in \textcolor{lightdustypink}{pink} indicate a \textbf{high positive correlation} ($>0.3$), while cells in \textcolor{lightdustygreen}{green} indicate a \textbf{high negative correlation} ($<-0.3$).}
    \label{tab:correlation_fs-units}
\end{table}
\begin{table}[H]
\small
    \centering
    \begin{tabular}{l|ccc}
        \hline
        Level & $R_{a1}$ & $R_{a2}$ & $R_{a3}$ \\ \hline
        & \multicolumn{3}{c}{Free-Form (FF)} \\ 
        \hline
$Q_1$ & \cellcolor{lightdustygreen} -0.502 & \cellcolor{lightdustygreen} -0.598 & \cellcolor{lightdustygreen} -0.529 \\
$Q_2$ & \cellcolor{lightdustygreen} -0.465 & \cellcolor{lightdustygreen} -0.3 & -0.174 \\ 
$Q_3$ & -0.232 & -0.181 & -0.079 \\ 
        
        \hline
                & \multicolumn{3}{c}{Multiple Choice (MC)} \\ \hline

                
$Q_1$ & \cellcolor{lightdustygreen} -0.528 & -0.023 & \cellcolor{lightdustygreen} -0.523 \\ 
$Q_2$ & 0.015 & -0.091 & -0.016 \\
$Q_3$ & -0.127 & 0.013 & -0.242 \\

        \hline
    \end{tabular}
    \caption{Correlation between model performance before redefinition with the percentage of anchored answers for each type of unit of measure redefinition and question level in CoT setup. 
    Cells highlighted in \textcolor{lightdustypink}{pink} indicate a \textbf{high positive correlation} ($>0.3$), while cells in \textcolor{lightdustygreen}{green} indicate a \textbf{high negative correlation} ($<-0.3$).}
    \label{tab:correlation_cot-units}
\end{table}


\subsubsection{Άρνηση Απόκρισης}

Παρουσιάζει ιδιαίτερο ενδιαφέρον το γεγονός ότι στην περίπτωση των επαναορισμών μονάδων μέτρησης το φαινόμενο της άρνησης απόκρισης είναι σχεδόν ανύπαρκτο. Μόνο τα μοντέλα της οικογένειας Mistral κατέγραψαν τέτοιες αρνήσεις, αλλά ακόμα και αυτές χαρακτήριζαν μεμονωμένα περιστατικά και όχι συστηματική συμπεριφορά. Η έντονη αυτή διαφορά σε σχέση με τις σταθερές δείχνει ότι το φαινόμενο σχετίζεται άμεσα με τον τρόπο με τον οποίο τα ΜΓΜ εσωτερικεύουν κάθε γνωσιακό πεδίο. Οι μονάδες μέτρησης φαίνεται να είναι λιγότερο "άκαμπτα" εσωτερικευμένες σε σχέση με τις επιστημονικές σταθερές.