\section{Συμπεράσματα}

Στην παρούσα εργασία μελετήσαμε εκτενώς την εργασία του επαναορισμού (redefinition), εξετάζοντας πώς τα Μεγάλα Γλωσσικά Μοντέλα (ΜΓΜ) αντιδρούν όταν τους παρουσιάζονται τροποποιημένες τιμές γνωστών επιστημονικών σταθερών και μονάδων μέτρησης. Στόχος μας ήταν να αξιολογήσουμε την ευελιξία τους έναντι της τάσης να προσκολλώνται στην εδραιωμένη γνώση. Τα ευρήματά μας αναδεικνύουν σημαντικά πρότυπα συμπεριφοράς των ΜΓΜ, φανερώνοντας περιορισμούς, οι οποίοι, μάλιστα, γίνονται εντονότεροι όσο αυξάνεται το μέγεθος των μοντέλων. Παρατηρούμε ότι, παρόλο που τα μεγαλύτερα μοντέλα εμφανίζουν ισχυρότερες ικανότητες συλλογιστικής υπό κανονικές συνθήκες, δυσκολεύονται περισσότερο όταν καλούνται να ακολουθήσουν επαναορισμένες τιμές, καθώς τείνουν να επιμένουν σε αυτές που έχουν απομνημονεύσει κατά την προεκπαίδευση. Εκτός αυτού, διαπιστώνουμε ότι παρουσιάζουν ψευδή αυτοπεποίθηση, προτιμώντας να απαντήσουν από το να απέχουν, ακόμα και όταν αυτό οδηγεί σε λάθη. 

Επιπλέον, τα πειράματά μας καλύπτουν ένα ευρύ φάσμα συνθηκών που αποσκοπούν στη δοκιμή της προσαρμοστικότητας των μοντέλων. Δημιουργήσαμε σύνολα δεδομένων με τύπους και επέπεδα επαναορισμών, καθώς και βαθμούς δυσκολίας των ερωτήσεων. Ταυτόχρονα, αξιολογήσαμε την επίδραση διαφορετικών μορφών απάντησης και τεχνικών προτροπής. Τα αποτελέσματα δείχνουν ότι το φαινόμενο προσκόλλησης εντείνεται σημαντικά στη μορφή των πολλαπλών επιλογών. Οι τεχνικές προτροπής επηρεάζουν μερικώς, αλλά απέχουν από το να εξαλείψουν το πρόβλημα.

Συνολικά, η εργασία μας αναδεικνύει σημαντικές αδυναμίες στη συλλογιστική και τη προσαρμοστικότητα των ΜΓΜ, οι οποίες εντείνονται με την αύξηση του μεγέθους τους. Τονίζουμε, επίσης, τη σημασία της βαθύτερης κατανόησης της συμπεριφοράς αυτών των μοντέλων, όχι μόνο ως προς το τι μπορόυν να κάνουν, αλλά και πού και γιατί αποτυγχάνουν. Το πείραμα του επαναορισμού προσφέρει ένα χρήσιμο πλαίσιο για τη μελέτη της εύθραυστης ισορροπίας μεταξύ μεγέθους, λογικής και συμμόρφωσης σε οδηγίες, και εδραιωμένων γνώσεων, και ελπίζουμε να αποτελέσει βάση για μελλοντική έρευνα στη διερεύνηση της προσαρμοστικότητας και της ανθεκτικότητας των ΜΓΜ.

\vspace{\baselineskip}
\vspace{\baselineskip}

\textbf{Ισορροπία ανάμεσα στη Λογική και την Ανθεκτικότητα}

Η μελέτη μας αναδεικνύει μία αντισταθμιστική σχέση στον σχεδιασμό και τη λειτουργία των ΜΓΜ: όσο πιο αυστηρά παραμένει ένα μοντέλο προσκολλημένο στην προεκπαιδευμένη γνώση του, τόσο λιγότερο πρόθυμο είναι να ακολουθήσει εναλλακτικά σενάρια, ακόμα και αν αυτά είναι λογικά αποδεκτά. Αυτό ενισχύει την πραγματολογική του ακρίβεια, αλλά περιορίζει τη λογική του ευελιξία. Από την άλλη πλευρά, ένα μοντέλο που ακολουθέι μη συμβατικά πειράματα επιδεικνύει μεγαλύτερη προσαρμοστικότητα, όμως είναι πιο ευάλωτο σε παραπλανητικές ή κακόβουλες προτροπές. Η ηθική πρόκληση, λοιπόν, είναι να βρεθεί μία ισορροπία: πώς μπορούμε να σχεδιάσουμε ΜΓΜ που είναι ταυτόχρονα ευέλικτα και αξιόπιστα;