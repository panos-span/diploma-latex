\chapter{System Design and Methodology}
\label{ch:methodology}

% \minitoc

This chapter presents the design and implementation of our agentic LLM system for SemEval-2026 Task 10. We implement a two-stage agentic workflow: \textbf{Subtask 1 (S1)} extracts psycholinguistic marker spans, and \textbf{Subtask 2 (S2)} predicts conspiracy endorsement conditioned on the document and the extracted markers. The design separates (i) LLM-mediated semantic decisions (what to extract / how to interpret stance) from (ii) deterministic operations that require exactness (character offsets, lightweight text statistics).

The chapter is organized as follows. We first describe the task and dataset (Section~\ref{sec:dataset}), then present the system architecture for both subtasks (Sections~\ref{sec:s1-system}--\ref{sec:s2-system}), the psycholinguistic grounding and marker ontology (Section~\ref{sec:psycholinguistic-method}), the prompt design and optimization methodology (Section~\ref{sec:prompts-methodology}), the models and tools used (Section~\ref{sec:models}), the experimental setup (Section~\ref{sec:exp-setup}), and finally the development trajectory documenting the system's evolution across six phases (Section~\ref{sec:development-trajectory}).

\section{Task Description and Dataset}
\label{sec:dataset}

\subsection{Task Description}

The SemEval-2026 Task 10 dataset comprises 4,800 annotations spanning 4,100 unique Reddit submission statements from more than 190 subreddits, divided into two subtasks:

\textbf{\textit{i)} S1: Conspiracy Marker Extraction} contains textual spans that express core conspiracy markers grounded in evolutionary psychology. One or more marker types may appear in each comment, falling in the following categories:
\begin{itemize}
  \item \textsc{Actor}: mentions of individual or group agents
  \item \textsc{Action}: descriptions of what the actor is doing
  \item \textsc{Effect}: consequences of the actions
  \item \textsc{Victim}: who is being harmed
  \item \textsc{Evidence}: claims or proof used to support the theory
\end{itemize}

\textbf{\textit{ii)} S2: Conspiracy Detection} assigns conspiracy-related or not conspiracy-related labels to Reddit comments.

\subsection{Data Preprocessing}

Since individual documents may have multiple annotators, we apply \textbf{majority-vote consensus} at both document and span level. For document labels, the most frequent annotation is selected and exact ties are discarded. For spans, overlapping annotations of the same marker type are clustered by character overlap; clusters reaching the majority threshold (over half of annotators) produce a single representative span (the longest in the cluster), while sub-threshold clusters are dropped. This yields deterministic, high-agreement annotations suitable for both training and few-shot retrieval.

After consensus, we remove near-duplicate documents via locality-sensitive hashing (LSH, 8 bands), reducing the training set from 3,682 rehydrated documents to 3,271 unique instances. \textit{Can't Tell} documents (607 in training, $\sim$18.6\%) are handled asymmetrically: they are \textbf{retained for S1} (marker extraction can still learn from ambiguous texts containing valid spans) but \textbf{excluded from S2} (conspiracy detection requires a binary ground truth). Additionally, documents with no annotated spans and no annotator disagreement are included in the S1 training corpus with 15\% probability, serving as negative calibration examples that teach the Generator to produce empty extractions for non-conspiratorial text.

For S2 corpus curation, a subtype-stratified sampling strategy selects documents across six rhetorical subtypes (hard negatives, mundane negatives, debunking negatives, evangelist conspiracy, insider conspiracy, and general conspiracy) to ensure balanced exposure during prompt optimization, with hard negatives defined broadly to include both non-conspiratorial texts containing markers \textit{and} texts matching debunking-vocabulary cues.

\subsection{Exploratory Data Analysis}
\label{sec:eda}

\subsubsection{Annotation Coverage}

As mentioned in the official website of the task, there are more than 4,100 unique Reddit comments, including 4,800 annotations in total. Most comments ($\sim$3,500) have only one annotation, 550 have two, and 50 have more. Regarding marker density, around 4,000 comments have at least one psycholinguistic marker annotation. The exact distribution of marker category coverage in comments is demonstrated in Figure~\ref{fig:num-of-markers}.

\begin{figure}[H]
  \centering
  \includegraphics[width=0.65\linewidth]{images/num-of-markers.png}
  \caption{Number of marker types in the dataset.}
  \label{fig:num-of-markers}
\end{figure}

\subsubsection{Label Distribution}

The dataset considers two clear classes, \textit{Yes (Conspiracy)} and \textit{No (Not Conspiracy)}, while the class \textit{Can't Tell} covers uncertain instances. The distribution of labels in the training data is illustrated in Figure~\ref{fig:label-distr}.

\begin{figure}[H]
  \centering
  \includegraphics[width=0.65\linewidth]{images/label-distr.png}
  \caption{Label distribution for conspiracy detection.}
  \label{fig:label-distr}
\end{figure}

Each marker category (\textit{Actor}, \textit{Action}, \textit{Effect}, \textit{Evidence}, \textit{Victim}) appears with different frequency within the dataset. The distribution of the five psycholinguistic marker types in the training dataset follows that of Figure~\ref{fig:marker-types}. Based on this Figure, we can conclude that conspiracy narratives rely on a small set of recurring rhetorical functions instantiated as markers, but no single function dominates the discourse.

\begin{figure}[H]
  \centering
  \includegraphics[width=0.65\linewidth]{images/marker-types.png}
  \caption{Frequency per marker type.}
  \label{fig:marker-types}
\end{figure}

\subsubsection{Annotation Density}

Annotation density is an interesting feature that implicitly indicates the difficulty of annotating the dataset: a sparsely annotated dataset showcases that conspiratorial evidence is semantically well-diffused within the text and hard to be acknowledged by humans. Indeed, several documents contain 0 annotations, while most documents do not exceed 20 annotations. The long-tailed distribution of markers per document presented in Figure~\ref{fig:density} validates the difficulty of the task.

\begin{figure}[H]
  \centering
  \includegraphics[width=0.65\linewidth]{images/annot-density.png}
  \caption{Number of marker annotations per document.}
  \label{fig:density}
\end{figure}

It is also useful to display the co-occurrences of markers in the training data, as in Figure~\ref{fig:cooccurrence}, indicating that marker types frequently appear together within the same documents, which in turn suggests that annotations capture recurring combinations of rhetorical roles.

\begin{figure}[H]
  \centering
  \includegraphics[width=0.65\linewidth]{images/cooccurrence.png}
  \caption{Marker type co-occurrences.}
  \label{fig:cooccurrence}
\end{figure}

The high self-co-occurrence of \textit{Action} and \textit{Actor} markers indicates that many documents describe multiple actions and multiple agents, consistent with narratives that unfold through sequences of events involving several entities rather than isolated claims. The strong co-occurrence between \textit{Action} and \textit{Actor} markers further highlights agency attribution as a central organizing principle, with conspiracy narratives frequently linking actors to specific actions. In contrast, \textit{Effect} and \textit{Victim} markers show more moderate self-co-occurrence, suggesting that while consequences and affected parties are recurrent elements, they are typically less elaborated than agency and action. Notably, \textit{Evidence} and \textit{Victim} markers rarely co-occur within the same documents, indicating a separation between evidential and victim-centered framing.

\subsubsection{Span Overlap Analysis}

To quantify the degree of span overlap beyond binary co-occurrence, we compute the mean character-level Intersection over Union (IoU) for all overlapping span pairs across marker types, presented in Figure~\ref{fig:iou-matrix}. The highest pairwise overlap occurs between \textsc{Actor} and \textsc{Victim} (mean IoU$=$0.65), reflecting the frequent rhetorical pattern where the accused party is simultaneously framed as the antagonist and the affected entity. \textsc{Action}$\leftrightarrow$\textsc{Effect} overlaps are also substantial (mean IoU$=$0.56), confirming that annotators sometimes struggle to delineate where a described process ends and its consequence begins. These overlap patterns directly motivate the S1 Critic's boundary enforcement rules.

\begin{figure}[H]
  \centering
  \includegraphics[width=0.65\linewidth]{images/mean-iou-matrix.png}
  \caption{Mean IoU of overlapping spans across marker type pairs. Higher values indicate greater boundary ambiguity between categories.}
  \label{fig:iou-matrix}
\end{figure}

\subsubsection{Marker Distribution Across Subreddits}

To further decompose the annotation density problem, we investigate the percentage of annotated markers per subreddit, illustrated in Figure~\ref{fig:subreddits}. Subreddits pose some noticeable differences regarding the dominant marker type. For example, \textit{Action} appears rather stable across subreddits, consistently describing \textit{what is being done}, regardless of community. The role of \textit{Actor} becomes more prominent in some communities (Israel\_Palestine) over other rhetorical roles. \textit{Evidence} presents some mild variability, and \textit{Victim}, associated with moralization and emotional appeal, covers higher proportion of markers in PlanetToday and TrueCrime subreddits.

\begin{figure}[H]
  \centering
  \includegraphics[width=0.65\linewidth]{images/pct-subreddit.png}
  \caption{Marker type distribution across Subreddits.}
  \label{fig:subreddits}
\end{figure}

\subsubsection{Span Position Analysis}

Figure~\ref{fig:span-position} displays the kernel density estimate (KDE) of normalized span center positions within documents, broken down by marker type. \textsc{Actor} spans concentrate toward the beginning of documents (median position$=$0.09), consistent with narrative openings that establish agency. In contrast, \textsc{Effect} spans peak later (median position$=$0.43), reflecting their role as narrative consequences that follow causal chains. \textsc{Evidence} spans exhibit the broadest positional spread, appearing throughout documents as authors interleave claims with supporting citations. These positional priors informed the S1 Generator's attention allocation.

\begin{figure}[H]
  \centering
  \includegraphics[width=0.65\linewidth]{images/span-position-analysis.png}
  \caption{Normalized position of marker spans within documents (0$=$start, 1$=$end). KDE per marker type.}
  \label{fig:span-position}
\end{figure}

\subsubsection{Span Length and Mass}

Analysis of marker span lengths (Figure~\ref{fig:span-length}) shows that most annotations correspond to short to medium-length text segments, while very long spans (more than 200 characters) are extremely rare. This indicates that the rhetorical roles captured by the annotation scheme are typically expressed through localized and well-defined linguistic units.

\begin{figure}[H]
  \centering
  \includegraphics[width=0.65\linewidth]{images/span-length.png}
  \caption{Span length distribution.}
  \label{fig:span-length}
\end{figure}

We also measure the ``span mass'' (Figure~\ref{fig:mass}), which reveals how much of the document is covered by annotated psycholinguistic spans (in characters). The relationship between total marker span length and the number of markers per document exhibits a clear positive trend, indicating that annotation coverage scales approximately linearly with annotation density.

\begin{figure}[H]
  \centering
  \includegraphics[width=0.65\linewidth]{images/span-mass.png}
  \caption{Marker span mass.}
  \label{fig:mass}
\end{figure}

\subsubsection{EDA-Driven Design Decisions}

Beyond descriptive statistics, our exploratory analysis produced quantitative insights that directly informed architectural choices:
\begin{itemize}
  \item Pairwise IoU analysis revealed that \textsc{Action}$\leftrightarrow$\textsc{Effect} spans overlap 46.4\% of the time at IoU$\geq$0.5, motivating the S1 Critic's explicit boundary enforcement.
  \item Pronoun density analysis showed that conspiracy texts use third-person distancing pronouns (\textit{they/them}) at significantly higher rates, informing the Forensic Profiler's Agency Gap metric.
  \item Question density analysis identified elevated rhetorical question rates in conspiracy texts, leading to the JAQing detection feature.
  \item Mann--Whitney tests with Benjamini--Hochberg correction confirmed that absolutist language rates differ significantly between conspiracy and non-conspiracy documents ($p_{\text{adj}}<0.001$, Cliff's $\delta=0.05$), validating the inclusion of epistemic intensity as a forensic profiler feature.
\end{itemize}

\subsubsection{Lexical Signal Analysis}
\label{sec:lexical-signals}

To quantify the psycholinguistic divergence between document classes, we compute per-document absolutist language and hedging rates (tokens per 1,000 words), aggregated by the three document-level labels. An \textit{absolutist} word is one expressing all-or-nothing thinking (e.g., ``always'', ``never'', ``completely'', ``nothing''), while a \textit{hedge} word signals epistemic uncertainty (e.g., ``perhaps'', ``maybe'', ``somewhat'', ``could'').

Table~\ref{tab:abs-hedge} reports means, medians, and standard deviations. Conspiracy-labeled documents exhibit a higher mean absolutist rate (0.484 per 1,000 words) compared to non-conspiracy (0.305) and can't-tell (0.337) documents. In contrast, hedging rates are relatively stable across all three labels, suggesting that hedging language is a general rhetorical feature rather than a conspiracy-specific signal.

\begin{table}[H]
  \centering
  \caption{Absolutist and hedging language rates per 1,000 words, by document label.}
  \label{tab:abs-hedge}
  \small
  \begin{tabular}{l|cccc|cccc}
    \hline
    \textbf{Label} & \multicolumn{4}{c|}{\textbf{Absolutist (per 1k)}} & \multicolumn{4}{c}{\textbf{Hedges (per 1k)}}                                              \\
                   & Mean                                              & Med.                                         & Std   & $n$  & Mean  & Med. & Std   & $n$  \\
    \hline
    Conspiracy     & 0.484                                             & 0.0                                          & 1.431 & 1169 & 0.610 & 0.0  & 1.430 & 1169 \\
    Non-conspiracy & 0.305                                             & 0.0                                          & 1.038 & 1595 & 0.667 & 0.0  & 1.592 & 1595 \\
    Can't Tell     & 0.337                                             & 0.0                                          & 1.264 & 607  & 0.679 & 0.0  & 1.729 & 607  \\
    \hline
  \end{tabular}
\end{table}

We further report effect sizes via Cliff's $\delta$ with Benjamini--Hochberg-corrected $p$-values in Table~\ref{tab:lexical-effects}. The conspiracy-vs-non-conspiracy comparison yields a small but statistically significant effect ($\delta=0.051$, $p_{\text{BH}}=0.0002$), confirming that absolutist framing is a genuine---if subtle---linguistic marker of conspiracy endorsement.

\begin{table}[H]
  \centering
  \caption{Pairwise effect sizes for absolutist language rates (Mann--Whitney, BH-corrected).}
  \label{tab:lexical-effects}
  \small
  \begin{tabular}{ll|cc}
    \hline
    \textbf{Group A} & \textbf{Group B} & \textbf{Cliff's $\delta$} & \textbf{$p_{\text{BH}}$} \\
    \hline
    Can't Tell       & Conspiracy       & $-0.049$                  & 0.008                    \\
    Can't Tell       & Non-conspiracy   & $+0.003$                  & 0.860                    \\
    Conspiracy       & Non-conspiracy   & $+0.051$                  & 0.0002                   \\
    \hline
  \end{tabular}
\end{table}

Figures~\ref{fig:absolutist-rate} and~\ref{fig:hedge-rate} visualize the distribution of these lexical signals across document labels, illustrating the right-tailed distribution characteristic of these low-density features.

\begin{figure}[H]
  \centering
  \includegraphics[width=0.65\linewidth]{images/absolutist_language_rate.png}
  \caption{Absolutist language rate per 1,000 words by document label. Conspiracy-labeled documents show elevated absolutist framing.}
  \label{fig:absolutist-rate}
\end{figure}

\begin{figure}[H]
  \centering
  \includegraphics[width=0.65\linewidth]{images/hedges_rate.png}
  \caption{Hedging language rate per 1,000 words by document label. Hedging rates remain stable across labels.}
  \label{fig:hedge-rate}
\end{figure}

\subsubsection{Label Coverage and Span Density}
\label{sec:label-coverage}

Table~\ref{tab:label-coverage} reports the coverage rate (fraction of conspiracy-labeled documents containing at least one span of that type) and the average number of spans per document, computed from the processed training set. \textsc{Actor} and \textsc{Action} markers appear in nearly 90\% of conspiracy documents, confirming their role as the primary building blocks of conspiratorial narratives. In contrast, \textsc{Victim} markers appear in only 59\% of documents, reflecting the fact that not all conspiracy narratives explicitly identify victims.

\begin{table}[H]
  \centering
  \caption{Label coverage rates and average spans per document in the processed training corpus.}
  \label{tab:label-coverage}
  \small
  \begin{tabular}{l|cc}
    \hline
    \textbf{Marker type} & \textbf{Coverage rate} & \textbf{Avg. spans/doc} \\
    \hline
    Actor                & 0.896                  & 2.023                   \\
    Action               & 0.887                  & 1.525                   \\
    Effect               & 0.720                  & 1.446                   \\
    Evidence             & 0.703                  & 1.442                   \\
    Victim               & 0.591                  & 1.575                   \\
    \hline
  \end{tabular}
\end{table}

\subsubsection{Pairwise Overlap Statistics}

Table~\ref{tab:overlap-stats} provides a comprehensive view of pairwise overlap statistics across all ten unordered marker-type pairs, computed from overlapping span pairs in the training corpus. The highest mean IoU occurs for \textsc{Actor}$\leftrightarrow$\textsc{Victim} (0.654), consistent with the rhetorical pattern where the accused party is simultaneously framed as antagonist and affected entity. The highest volume of overlapping pairs involves \textsc{Effect}$\leftrightarrow$\textsc{Victim} ($n=227$), reflecting the narrative tendency to co-locate consequences with affected parties.

\begin{table}[H]
  \centering
  \caption{Pairwise overlap statistics for all marker-type combinations. \textit{IoU@0.5}: fraction of pairs with IoU$\geq$0.5.}
  \label{tab:overlap-stats}
  \small
  \begin{tabular}{l|rccc}
    \hline
    \textbf{Pair}   & $n$ & \textbf{Mean IoU} & \textbf{IoU@0.5} & \textbf{Contain rate} \\
    \hline
    Actor/Victim    & 77  & 0.654             & 0.610            & 0.935                 \\
    Action/Effect   & 207 & 0.556             & 0.464            & 0.879                 \\
    Action/Evidence & 193 & 0.421             & 0.363            & 0.829                 \\
    Effect/Evidence & 134 & 0.416             & 0.328            & 0.873                 \\
    Effect/Victim   & 227 & 0.292             & 0.159            & 0.934                 \\
    Action/Actor    & 132 & 0.294             & 0.189            & 0.864                 \\
    Action/Victim   & 199 & 0.289             & 0.181            & 0.960                 \\
    Actor/Evidence  & 176 & 0.258             & 0.153            & 0.903                 \\
    Evidence/Victim & 116 & 0.218             & 0.112            & 0.922                 \\
    Actor/Effect    & 69  & 0.186             & 0.058            & 0.884                 \\
    \hline
  \end{tabular}
\end{table}

The containment rate---the fraction of pairs where one span is entirely contained within the other---exceeds 85\% for all marker combinations. This high containment reflects the nested nature of conspiratorial narratives, where shorter spans (typically \textsc{Actor}) are embedded within longer spans describing \textsc{Action} or \textsc{Effect}. These statistics directly motivated the S1 Critic's boundary enforcement rules and the Deterministic Verifier's handling of overlapping span candidates.


\section{System Architecture}
\label{sec:system-overview}

Figure~\ref{fig:arch} summarizes the inference flow of our two-stage agentic system.

\begin{figure*}[t]
    \vspace{-6pt}
    \centering
    \scriptsize
    \begin{tikzpicture}[
        node distance=0.6cm and 0.4cm,
        >={Stealth[length=3pt]},
        % Styles
        inputbox/.style={rectangle, draw, rounded corners=2pt, minimum height=0.7cm,
                minimum width=1.4cm, align=center, font=\scriptsize\sffamily, fill=gray!10},
        ragbox/.style={rectangle, draw, rounded corners=2pt, minimum height=0.7cm,
                minimum width=1.4cm, align=center, font=\scriptsize\sffamily, fill=green!10},
        llmbox/.style={rectangle, draw, rounded corners=2pt, minimum height=0.7cm,
                minimum width=1.4cm, align=center, font=\scriptsize\sffamily, fill=blue!10},
        detbox/.style={rectangle, draw, rounded corners=2pt, minimum height=0.7cm,
                minimum width=1.4cm, align=center, font=\scriptsize\sffamily, fill=orange!15},
        outputbox/.style={rectangle, draw, rounded corners=2pt, minimum height=0.7cm,
                minimum width=1.4cm, align=center, font=\scriptsize\sffamily, thick},
        arrow/.style={->, thick, black!70, rounded corners=2pt}, dasharrow/.style={->,
                thick, dashed, black!50, rounded corners=2pt},
        grouplab/.style={font=\bfseries\scriptsize\sffamily, text=black!70},
        labeltext/.style={font=\tiny\sffamily, text=black!60, midway, fill=white, inner
                sep=1pt}, ]

        % === INPUT ===
        \node[inputbox] (input) {Document};

        % === S1 PIPELINE (Top Row) ===
        \node[ragbox, right=0.8cm of input] (s1rag) {Stratified\\Few-shot\\Retrieval};
        \node[llmbox, right=0.5cm of s1rag] (gen) {DD-CoT\\Generator};
        \node[llmbox, right=0.5cm of gen] (critic) {Enhanced\\Critic};
        \node[llmbox, right=0.5cm of critic] (refiner) {Refiner\\Agent};
        \node[detbox, right=0.5cm of refiner] (verifier) {Deterministic\\Verifier};
        \node[outputbox, right=0.5cm of verifier] (s1out) {S1 Spans};

        % S1 Connections
        \draw[arrow] (input) -- (s1rag);
        \draw[arrow] (s1rag) -- (gen);
        \draw[arrow] (gen) -- (critic);
        \draw[arrow] (critic) -- (refiner);
        \draw[arrow] (refiner) -- (verifier);
        \draw[arrow] (verifier) -- (s1out);

        % S1 Group Label
        \begin{scope}[on background layer]
            \node[draw=blue!40, fill=blue!3, rounded corners=4pt, dashed,
            fit=(s1rag)(gen)(critic)(refiner)(verifier)(s1out),
            inner sep=8pt, label={[grouplab]above:{S1: Marker Extraction (Self-Refine + DD-CoT)}}] (s1group) {};
        \end{scope}

        % === S2 PIPELINE (Bottom Row) ===
        \node[ragbox, below=2.2cm of s1rag] (s2rag) {Contrastive\\Few-shot\\Retrieval};
        \node[detbox, right=0.5cm of s2rag] (forensic) {Forensic\\Profiler};

        % PARALLEL COUNCIL (Strict 2x2 Grid)
        \node[llmbox, right=0.8cm of forensic, yshift=0.6cm] (pros) {Prosecutor};
        \node[llmbox, right=0.5cm of pros] (literal) {Literalist\\{\tiny $\tau$=0.4}};

        \node[llmbox, right=0.8cm of forensic, yshift=-0.6cm] (defense) {Defense};
        \node[llmbox, right=0.5cm of defense] (profiler) {Stance\\Profiler};

        % Judge
        \node[llmbox, right=0.6cm of $(literal.east)!0.5!(profiler.east)$] (judge) {Calibrated\\Judge};
        \node[outputbox, right=0.5cm of judge] (verdict) {Final\\Verdict};

        % Fork (Clean Distribution)
        \coordinate[right=0.4cm of forensic] (fork_point);
        \draw[thick, black!70] (forensic.east) -- (fork_point);
        \draw[arrow] (fork_point) |- (pros.west);
        \draw[arrow] (fork_point) |- (defense.west);

        % S2 Connections
        \draw[arrow] (input.south) |- (s2rag.west);
        \draw[arrow] (input.south) |- ($(s2rag.south) + (0,-0.5)$) -| (forensic.south);

        % S1 Out to Council 
        \draw[arrow] (s1out.south) -- ++(0,-1.5) -| (fork_point);

        % S2RAG to Council (Bypassing Forensic)
        \draw[arrow] (s2rag.north) -- ++(0,0.3) -| (fork_point);

        % Internal Flow
        \draw[arrow] (pros) -- (literal);
        \draw[arrow] (defense) -- (profiler);

        % Merge (Clean Join)
        \draw[arrow] (literal.east) -- ++(0.2,0) |- (judge.west);
        \draw[arrow] (profiler.east) -- ++(0.2,0) |- (judge.west);
        \draw[arrow] (judge) -- (verdict);

        % S2 Group Label
        \begin{scope}[on background layer]
            \node[draw=red!40, fill=red!3, rounded corners=4pt, dashed,
            fit=(s2rag)(forensic)(pros)(defense)(literal)(profiler)(judge)(verdict),
            inner sep=8pt, label={[grouplab]below:{S2: Anti-Echo Chamber (Parallel Council)}}] (s2group) {};
        \end{scope}

    \end{tikzpicture}
    \vspace{-4pt}
    \caption{System architecture overview. \textbf{S1} (top): DD-CoT Self-Refine extracts marker strings, then a deterministic verifier anchors them to character offsets. \textbf{S2} (bottom): an Anti-Echo Chamber (forensic profiling + parallel council + calibrated judge) predicts conspiracy endorsement.}
    \label{fig:arch}
    \vspace{-10pt}
\end{figure*}

\subsection{S1: Marker Extraction via DD-CoT}
\label{sec:s1-system}

S1 produces a set of labeled spans by combining a self-refinement loop with a deterministic span locator. The graph consumes the document and retrieved few-shot precedents, generates candidate marker \emph{strings} with labels, iteratively corrects them, then anchors each string to \texttt{startIndex} and \texttt{endIndex} in the original text.

\subsubsection{Hybrid Architecture Rationale}

We explicitly decouple \textit{semantic identification} from \textit{span indexing}. LLMs can justify category assignments but are brittle at character-accurate localization \cite{fu2024struggle}. We therefore (i) ask the LLM to emit verbatim marker strings with labels, and (ii) compute offsets with a deterministic locator that performs exact matching against the source text. This avoids ``hallucinated spans'' and off-by-one indices while preserving the LLM's interpretive signal \cite{ogasa-arase-2025-hallucinated}.

\subsubsection{Dynamic Discriminative Chain-of-Thought (DD-CoT)}

DD-CoT extends CoT reasoning \cite{wei2022chain} with an explicit \emph{discrimination step}. For each candidate span, the generator must state (i) evidence for the chosen label and (ii) a short counter-argument against at least one confusable label. This forces the model to commit to a decision boundary in frequent confusions (e.g., \textsc{Actor} vs. \textsc{Victim}, \textsc{Action} vs.\ \textsc{Effect}) rather than producing post-hoc rationales.

\subsubsection{Agent Pipeline}

The Self-Refine loop comprises four sequential nodes:
\begin{enumerate}
    \item A DD-CoT \textbf{Generator} that proposes labeled marker strings;
    \item An \textbf{Enhanced Critic} that checks verbatimness, boundaries, label discrimination, and missing spans;
    \item A \textbf{Refiner} that applies minimal edits; and
    \item A \textbf{Deterministic Verifier} that maps strings to character offsets and deduplicates overlaps.
\end{enumerate}

The Self-Refine pattern follows the standard critique--revise loop \cite{madaan2023selfrefine} but operates over typed intermediate artifacts (candidate spans, critiques, and edits), improving controllability and enabling deterministic verification.

\subsubsection{Deterministic Verifier}

The Deterministic Verifier is a non-LLM post-processing node that serves as the \textit{structural locator}, anchoring LLM-generated text strings to character-precise offsets through a five-tier matching cascade:
\begin{enumerate}
    \item[(i)] \textbf{Exact match:} Byte-for-byte substring search supporting nth-occurrence disambiguation.
    \item[(ii)] \textbf{Case-insensitive:} Unicode-safe lowered comparison with original-position projection.
    \item[(iii)] \textbf{Normalized:} Smart-quote straightening, whitespace collapse, and lowering with index remapping to recover original character offsets.
    \item[(iv)] \textbf{Fuzzy (Levenshtein):} Approximate matching with maximum edit distance $\leq 15\%$ of snippet length (minimum~1), activated only for spans $>$4 characters to avoid spurious short matches.
    \item[(v)] \textbf{SequenceMatcher alignment:} LCS-based last-resort recovery requiring $\geq 60\%$ character coverage and compactness $\leq 1.5\times$ snippet length, with word-boundary snapping.
\end{enumerate}
Each tier is attempted in order; the first successful match is accepted. Additionally, the Verifier implements aggressive cross-label deduplication to eliminate overlapping or duplicate spans.

\subsection{S2: Classification via Anti-Echo Chamber}
\label{sec:s2-system}

S2 targets a specific failure mode: \textbf{Reporter Trap} false positives, in which topical discussion of conspiracies is conflated with endorsement (e.g., reporting, debunking, satire). Single-pass classifiers often over-commit early and underweight stance cues \cite{wan-etal-2025-unveiling}. We therefore structure S2 as: (i) a deterministic \emph{Forensic Profiler} that emits stance-relevant warnings; (ii) a \emph{Parallel Council} that produces independent pro/contra analyses; and (iii) a \emph{Calibrated Judge} that aggregates votes with conservative confidence rules.

\subsubsection{Forensic Profiler}

Before LLM deliberation, a deterministic node computes lightweight linguistic signals that are injected as structured warnings. Six metrics are extracted, each normalized by total word count $|W|$:
\begin{enumerate}
    \item \textbf{Attribution Density:} $\text{AD} = |\{w \in W : w \in V_{\text{attr}}\}| \,/\, |W|$, where $V_{\text{attr}}$ includes distancing verbs (\textit{said}, \textit{claimed}, \textit{according to}, \textit{reported}, \textit{sources}). Texts with AD $> 3.5\%$ receive an explicit \textsc{Reporter\_Warning} flag.
    \item \textbf{Shouting Score:} $\text{SS} = |\{w \in W : w = \texttt{UPPER}(w) \land |w| > 1\}| \,/\, |W|$. Scores exceeding $10\%$ trigger an \textsc{Emotional\_Intensity} flag.
    \item \textbf{JAQing Detection:} A boolean flag activated when question density $> 0.35$ (questions per sentence) \textit{and} hedging ratio $> 5\%$, identifying the ``Just Asking Questions'' rhetorical manipulation pattern.
    \item \textbf{Agency Gap:} Passive voice proxy computed as the frequency of passive-voice indicators (\textit{been}, \textit{being}, \textit{was}, \textit{were}, \textit{by}). Values $> 6\%$ suggest hidden agency attribution.
    \item \textbf{Epistemic Intensity:} Frequency of truth-claiming terms (\textit{proof}, \textit{truth}, \textit{exposed}, \textit{revealed}, \textit{undeniable}) normalized by $|W|$.
    \item \textbf{Question Density:} Number of question marks per sentence, used as a component of JAQing detection.
\end{enumerate}

These metrics are injected into the Calibrated Judge's case file as structured contextual warnings.

\subsubsection{Parallel Council Architecture}

The \textbf{Anti-Echo Chamber} enforces independent assessment by four personas that receive identical inputs (document, S1 markers, retrieval context, profiler warnings) and produce structured votes without seeing each other's outputs:
\begin{enumerate}
    \item \textbf{Prosecutor:} evidence \emph{for} endorsement.
    \item \textbf{Defense Attorney:} evidence \emph{against} endorsement (reporting/debunking/satire cues).
    \item \textbf{Literalist:} literal entailment and burden-of-proof checks.
    \item \textbf{Profiler:} stance cues (certainty, framing, group dynamics).
\end{enumerate}

Each juror outputs: (i) a verdict with confidence, (ii) the primary evidence supporting that verdict, (iii) a mandatory counter-argument, and (iv) uncertainty flags for known ambiguities. This design reduces information leakage and ordering effects typical of sequential debate.

\subsubsection{Calibrated Judge}

The Judge aggregates votes and applies conservative confidence rules. We compute a weighted score:
$$W = \sum_{j=1}^{4} \begin{cases} +c_j & \text{if } v_j = \texttt{conspiracy} \\ -c_j & \text{if } v_j = \texttt{non} \end{cases}$$
with equal persona weights, where $c_j \in [0,1]$ is juror confidence and $v_j$ its verdict. When the council splits ($2$--$2$), we cap confidence at $0.75$, mark the case \texttt{borderline}, and default to \texttt{non} when evidence remains ambiguous. Overrides of a non-split majority are explicitly flagged.

\subsection{Contrastive Few-shot Retrieval}
\label{sec:contrastive-rag}

In-context few-shot examples are retrieved for the given document (no augmentation is performed). A contrastive strategy is used to supply discriminative precedents (including hard negatives for the Reporter Trap).

\subsubsection{S1: Stratified Contrastive Sampling}

For marker extraction, we implement a dual-axis contrastive strategy. First, we retrieve \textbf{balanced positive and negative examples} to teach the model that psycholinguistic markers can appear in \textit{both} contexts. Second, within these retrieved examples, we apply \textbf{marker-type stratification}, allocating 60\% retrieval weight to underrepresented categories (\textsc{Evidence} and \textsc{Victim}). Finally, all candidates undergo \textbf{cross-encoder reranking} (BAAI/bge-reranker-v2-m3) \cite{bge_m3} to prioritize examples with similar discourse structure over mere lexical overlap.

The overall contrastive retrieval strategy is illustrated in Figure~\ref{fig:rag}.

\begin{figure}[t]
    \centering
    \small
    \resizebox{\linewidth}{!}{
        \begin{tikzpicture}[
            node distance=0.5cm and 0.4cm,
            >={Stealth[length=3pt]},
            box/.style={rectangle, draw, rounded corners=2pt, minimum height=0.6cm,
                    align=center, font=\scriptsize\sffamily, inner sep=2pt},
            sbox/.style={box, fill=blue!10, minimum width=2.2cm},
            nbox/.style={box, fill=red!10, minimum width=2.2cm},
            gbox/.style={box, fill=green!10, minimum width=2.8cm},
            arrow/.style={->, thick, black!70},
            ]

            \node[box, fill=gray!15, minimum width=2.5cm] (query) {Input Document};
            \node[box, fill=yellow!15, below=0.5cm of query, minimum width=2.5cm] (chroma) {ChromaDB Embeddings};
            \draw[arrow] (query) -- node[right, font=\tiny\sffamily] {embed} (chroma);

            \node[sbox, below left=0.6cm and 0.3cm of chroma] (s1ret) {Balanced + Type\\Stratification (60\%)};
            \node[nbox, below right=0.6cm and 0.3cm of chroma] (s2ret) {Hard Negatives\\(Non-CT + Markers)};
            \draw[arrow] (chroma.south) -- ++(-0.1,-0.2) -| (s1ret.north);
            \draw[arrow] (chroma.south) -- ++(0.1,-0.2) -| (s2ret.north);

            \node[gbox, below=0.5cm of s1ret] (rerank1) {Cross-Encoder\\Reranking (3$\times$)};
            \node[gbox, below=0.5cm of s2ret] (rerank2) {Reranking (4$\times$)\\+ Filtering};
            \draw[arrow] (s1ret) -- (rerank1);
            \draw[arrow] (s2ret) -- (rerank2);

            \node[box, fill=blue!5, below=0.5cm of rerank1, minimum width=2.2cm] (out1) {S1 Few-Shots};
            \node[box, fill=red!5, below=0.5cm of rerank2, minimum width=2.2cm] (out2) {S2 Precedents};
            \draw[arrow] (rerank1) -- (out1);
            \draw[arrow] (rerank2) -- (out2);

        \end{tikzpicture}
    }
    \caption{Contrastive few-shot retrieval architecture.}
    \label{fig:rag}
\end{figure}

\subsubsection{S2: Hard Negative Mining}

For conspiracy detection, we implement a \textbf{pure contrastive strategy} via hard negative mining \cite{karpukhin-etal-2020-dense}. Standard similarity-based retrieval causes the model to conflate \textit{topical similarity} with \textit{stance endorsement}. To explicitly teach the boundary, we retrieve documents labeled ``non-conspiracy'' \textit{that contain S1 markers}. These are \textit{hard} negatives because they share conspiracy-related vocabulary but differ in stance. By forcing the model to compare structurally similar examples with opposite labels, we compel it to attend to \textbf{stance cues} such as attribution verbs, hedging markers, and framing signals, rather than mere topic keywords. Candidates undergo the same cross-encoder reranking with an elevated overretrieve factor ($4\times$ vs.\ $3\times$ for S1).

\section{Psycholinguistic Grounding}
\label{sec:psycholinguistic-method}

A distinguishing feature of our approach is the explicit grounding in \textbf{evolutionary psychology} and \textbf{psycholinguistic theory} of conspiratorial rhetoric. Rather than treating conspiracy detection as a surface-level pattern matching task, we encode theoretical knowledge about \textit{why} conspiracy theories emerge and \textit{how} they manifest linguistically~\cite{douglas2019understanding}.


\subsection{Theoretical Foundation}

The core prompt preamble establishes the psycholinguistic framework that guides all extraction and classification agents:

\begin{tcolorbox}[colback=purple!5, colframe=purple!50!black, title=\textbf{Psycholinguistic Preamble}]
``You are an expert computational psycholinguist. Align your reasoning with psycholinguistic and evolutionary accounts of conspiratorial rhetoric.''
\end{tcolorbox}

This framing is not merely cosmetic---it primes the model to distinguish between genuine conspiratorial ideation and superficially similar discourse (policy criticism, news reporting, satire) by attending to the \textit{psychological function} of language rather than surface-level keyword matching.


\subsection{Conspiracy Marker Ontology}
\label{sec:marker-ontology}

The five psycholinguistic marker categories are grounded in evolutionary threat detection heuristics. Each marker corresponds to a component of the conspiratorial narrative structure:

\begin{enumerate}
    \item \textbf{Actor (The Conspirators):} Agents alleged to secretly orchestrate events. The Actor category encompasses three subtypes that reflect different levels of abstraction in conspiratorial attribution:
    \begin{itemize}
        \item \textit{Core Actors:} Vague collective agents serving as prototypical outgroup targets (``The Elites'', ``They'', ``Globalists''). These exploit the human cognitive bias toward agent detection in complex causal chains.
        \item \textit{Systemic Actors:} Abstract forces or institutions framed with malicious agency (``The Migration Act'', ``Big Pharma''). These conflate institutional processes with intentional malice.
        \item \textit{Institutional Actors:} Specific formal bodies presented as conspiratorial agents (``The CIA'', ``The Media''), regardless of whether they are praised or condemned.
    \end{itemize}
    
    \item \textbf{Action (The Mechanism):} What the Actor \textit{does}---verbs implying control, secrecy, or harm. The action vocabulary provides critical discriminative signal:
    \begin{itemize}
        \item \textit{Conspiratorial actions:} ``engineered'', ``suppressed'', ``brainwashed'', ``censored'', ``staged'', ``weaponized''
        \item \textit{Non-conspiratorial actions:} ``presented'', ``issued'', ``recommended'', ``held a meeting'', ``published''
    \end{itemize}
    
    \item \textbf{Effect (The Outcome):} The goal or anticipated consequence of the malicious Action. Effects typically invoke existential or large-scale threats: ``depopulation'', ``total control'', ``mass death'', ``enslavement'', ``loss of freedoms.''
    
    \item \textbf{Victim (The Target):} The entity bearing the ``evolutionary cost'' of the conspiracy. Victims are framed as innocents being exploited: ``our children'', ``the public'', ``taxpayers'', ``patients.'' The Victim category triggers evolutionary in-group protection instincts.
    
    \item \textbf{Evidence (Epistemic Weaponry):} Rhetorical supports used to validate conspiratorial claims. Unlike standard evidence citations, conspiracy ``evidence'' often exhibits distinctive patterns:
    \begin{itemize}
        \item \textit{Privileged knowledge claims:} ``The video'', ``The proof'', ``Hunter's Laptop'', ``Leaked files''
        \item \textit{Quantitative inflation:} ``A massive amount of data'', ``100s of cases''
        \item \textit{Self-sealing epistemics:} ``Do your own research'', ``Connect the dots'', ``Mainstream media lies''
    \end{itemize}
\end{enumerate}

Table~\ref{tab:marker-examples} provides concrete examples of how span extraction distinguishes between conspiratorial and non-conspiratorial instances of each marker category.

\begin{table}[H]
\centering
\caption{Marker category examples: conspiratorial vs. non-conspiratorial instances.}
\label{tab:marker-examples}
\begin{tabular}{@{}llp{5cm}p{5cm}@{}}
\toprule
\textbf{Category} & \textbf{Conspiratorial} & \textbf{Non-conspiratorial} \\
\midrule
Actor & ``the corrupt media'' & ``media'' (generic) \\
Actor & ``Big Pharma'' & ``pharmaceutical companies'' \\
Actor & ``the deep state'' & ``government agencies'' \\
Action & ``rigged the election'' & ``held a meeting'' \\
Action & ``suppressed evidence'' & ``presented data'' \\
Effect & ``total surveillance'' & ``increased monitoring'' \\
Victim & ``our children'' & ``participants in the study'' \\
Evidence & ``leaked emails'' & ``published report'' \\
\bottomrule
\end{tabular}
\end{table}


\subsection{Explicit Conspiratorial Signal Detection}

The S2 classification prompts encode \textbf{psychological signatures} of conspiracy ideation that each council member is trained to detect.

\subsubsection{Nefarious Intent Detection}

The Prosecutor prompt encodes the \textbf{Institutional Capture Rule}:

\begin{quote}
``If an institution is framed as: (a) Working against the people, (b) Controlled by hidden interests, (c) Deliberately betraying its mandate $\rightarrow$ CONSPIRACY. Distinction: Failure/incompetence $\rightarrow$ NON; Intentional cover-up $\rightarrow$ INDICT.''
\end{quote}

This distinguishes \textit{malice} from \textit{incompetence}---a critical boundary in conspiracy detection that the Hanlon's Razor defense specifically exploits.

\subsubsection{Epistemic Arrogance Markers}

The Profiler prompt detects ``True Believer'' signals that indicate strong personal identification with conspiratorial worldviews:

\begin{itemize}
    \item \textbf{Privileged insight claims:} ``Wake up'', ``Do your research'', ``You're being lied to'', ``The truth is coming out'' $\rightarrow$ HIGH-CONFIDENCE CONSPIRACY.
    \item \textbf{Us-vs-Them identity framing:} ``We'' = enlightened victims, ``They'' = hidden controllers.
    \item \textbf{Moral absolutism:} ``Evil'', ``demonic'', ``crimes against humanity.''
    \item \textbf{Ideological dog whistles:} ``Globalist'', ``Cabal'', ``Sheeple'', ``False Flag'' (unironic usage).
\end{itemize}

\subsubsection{Structural Assertion Rule}

All S2 prompts encode the \textbf{Structural Assertion Rule}---a critical insight from error analysis that addresses the non-obvious boundary between assertion and reporting:

\begin{tcolorbox}[colback=red!5, colframe=red!50!black, title=\textbf{Structural Assertion Rule (Critical Patch)}]
A conspiracy claim does NOT require first-person belief, emotional language, or secret plotting.

\textbf{Rule:} If the author \textit{asserts the existence of coordinated, intentional malice as a factual condition}---even impersonally or legally---this constitutes \textbf{ENDORSEMENT BY ASSERTION}.

\textbf{Examples (CONSPIRACY):}
\begin{itemize}[noitemsep]
    \item ``There has been a conspiracy to undermine...''
    \item ``The system was designed to keep people poor.''
    \item ``This operation was intended to suppress the truth.''
\end{itemize}

\textbf{Non-Examples (NON):}
\begin{itemize}[noitemsep]
    \item ``Some claim there was a conspiracy...'' (attributed)
    \item ``The article alleges...'' (reporting frame)
    \item ``Critics argue...'' (distancing)
\end{itemize}

\textbf{Important:} Passive voice and formal tone do NOT imply neutrality.
\end{tcolorbox}


\subsection{Data-Aware Context Encoding}

The prompts encode knowledge about the specific characteristics of the data source to prevent systematic errors from misinterpreting domain-specific conventions:

\begin{tcolorbox}[colback=gray!5, colframe=gray!50, title=\textbf{Data Profile Block}]
\begin{itemize}[noitemsep]
    \item \textbf{Source:} Reddit Submission Statements (SS)
    \item \textbf{Function:} SS is a comment required by moderators to explain a link
    \item \textbf{Implication:} Text often \textit{summarizes} linked content (``OP claims that...'') rather than expressing user's own belief
    \item \textbf{Rhetoric:} High prevalence of sarcasm, ``Just Asking Questions'' (JAQ), community slang (``based'', ``shill'', ``glowie'')
    \item \textbf{Structure:} Markdown-flattened; URLs replaced with [URL]
\end{itemize}
\end{tcolorbox}

This data-awareness is injected into all agent prompts, enabling them to correctly interpret Reddit-specific conventions such as Submission Statements (which summarize linked content without necessarily endorsing it) and community-specific slang.


\subsection{The Cues and Pitfalls Playbook}

Beyond formal definitions, prompts encode explicit \textbf{cues} (positive signals indicating conspiratorial content) and \textbf{pitfalls} (common error patterns to avoid):

\paragraph{Positive Cues:}
\begin{itemize}[noitemsep]
    \item \textbf{Vague/Collective Agents:} ``they'', ``the elite'', ``globalists'', ``deep state'', ``big pharma''
    \item \textbf{Control/Hostility Verbs:} plot, engineer, manipulate, gaslight, weaponize
    \item \textbf{Self-Sealing Epistemics:} ``do your own research'', ``connect the dots'', ``mainstream media lies''
\end{itemize}

\paragraph{Pitfalls to Avoid:}
\begin{itemize}[noitemsep]
    \item \textbf{The ``Reporter'' Trap:} Submission Statements often summarize linked articles. ``The article argues that...'' is NOT endorsement.
    \item \textbf{The ``JAQ'' Trap:} ``Just Asking Questions'' is a conspiracy tactic ONLY if the question presupposes a hidden plot.
    \item \textbf{Overt Tyranny Clause:} A conspiracy does NOT need to be hidden. If government/corporation is alleged to \textit{intentionally oppress} openly $\rightarrow$ still CONSPIRACY.
\end{itemize}

This comprehensive psycholinguistic grounding ensures that our system's classification decisions are both \textit{theory-informed} and \textit{empirically calibrated} against the specific failure modes identified during development.

\setcounter{secnumdepth}{4}

\section{Prompt Engineering}
\label{sec:prompting}

\subsection{Prompts and Prompt Engineering}

A \textit{prompt} serves as an input consisting of manually predefined instructions or cues provided to Large Language Models (LLMs) in order to guide their outputs on specific tasks \cite{promptreportsystematicsurvey}. The systematic practice of designing, structuring, and formulating these instructions in a specialized way that effectively steers model behavior toward desired responses is referred to as \textit{prompt engineering} \cite{promptreportsystematicsurvey} and, over the past few years, it has emerged as a key technique for enhancing LLM performance across a wide range of tasks and domains \cite{systematicsurveypromptengineering}. The significance of this new approach lies in its core advantage: unlike previous conventional methods such as re-training and fine-tuning, prompt engineering leverages the pre-existing knowledge encoded in the LLM to improve the generated output without altering its internal parameters \cite{surveypromptengineeringmethods}. This allows for flexible adaptation to new tasks while entirely avoiding time- and resource-intensive training procedures, thereby maintaining computational efficiency. However, despite its power, prompt engineering remains inherently brittle. LLMs display high sensitivity to the input prompt, which means that even slight changes in wording, the use of synonyms, capitalization, or spacing can yield substantial shifts in performance \cite{promptreportsystematicsurvey}. The choice of question format appears to deeply influence model behavior as well. For instance, forming "yes or no" or multiple choice questions often results in completely different outputs compared to simple unrestricted generation. In fact, even minor perturbations, like changing the order of the possible options are displayed in the multiple choice format, can affect results \cite{promptreportsystematicsurvey}. All of these highlight an intriguing challenge at the heart of prompting engineering: the careful search for the most appropriate prompt that can unlock this method's full potential and eventually achieve optimal LLM performance under the given task \cite{liu2021pretrainpromptpredictsystematic}.

\subsection{Prompt Templates}

To simplify interactions with LLMs and boost usability across specialized tasks, prompts are usually assembled using \textit{prompt templates} \cite{templates}. Prompt templates are structured input formats that typically function as parameterized instructions, containing one or multiple placeholders for variables that, during experimentation, are being replaced by specific textual--or other--instances to create finalized prompts \cite{promptreportsystematicsurvey}. In this way, the same instruction pattern can be systematically applied to a large volume of data, making it feasible to scale from testing a few examples to running large datasets efficiently.

Consider the task of sentiment analysis of tweets. Figure \ref{fig:template} includes an example of a prompt template that instructs models to classify a tweet as either positive or negative. In this template, \{TWEET\} is the variable placeholder that is replaced with the actual tweet to be analyzed, producing a prompt \textit{instance} which is then fed to the LLM for inference \cite{promptreportsystematicsurvey}.

\begin{figure}[H]
    \vskip -0.01in
    \centering
    \includegraphics[width=0.5\linewidth]{images/template.png}
    \caption{Prompt template example for the task of tweet sentiment analysis \cite{promptreportsystematicsurvey}.} \label{fig:template}
    \vskip -0.09in
\end{figure}



\subsection{Prompting Techniques}

In the search for the "most efficient prompt" that can optimally extract the desired response for a specific task, several \textit{prompting techniques} have been developed and evolved to improve the ability of Large Language Models to follow instructions and reason successfully.

\subsubsection{Zero-Shot Prompting}

Zero-Shot prompting (Figure \ref{fig:zero}) is the simplest form of prompt engineering, consisting solely of a direct instruction to complete a specific task, without providing additional examples or cues on how to approach it \cite{systematicsurveypromptengineering}. In this setup, the model relies on its embedded knowledge to generate predictions, which often proves sufficient to perform adequately on various downstream tasks, including reading comprehension, translation, or summarization, thanks to its extensive pre-training on vast amounts of data \cite{kojima2023largelanguagemodelszeroshot}. However, the Zero-Shot technique is typically outperformed, especially under more difficult scenarios that require nuanced understanding or complex reasoning (\cite{surveypromptengineeringmethods}; \cite{brown2020languagemodelsfewshotlearners}). Nevertheless, Zero-Shot prompting remains a foundational method, setting a baseline to compare with more advanced strategies.

\subsubsection{One-Shot Prompting}

The One-Shot prompting strategy (Figure \ref{fig:one}) includes a single example of successful performance on a specific instance of the described task, to help the model better understand the task's requirements, expected output format, or preferred reasoning process. This method is considered to be closer to the way more complex tasks are often communicated to humans, where the absence of a worked example usually leads to confusion about how to proceed \cite{brown2020languagemodelsfewshotlearners}.

\subsubsection{Few-Shot Prompting}

Few-Shot prompting (Figure \ref{fig:few}) operates exactly like one-Shot prompting, but instead of one, it provides multiple demonstrations of input-output instances to enhance the model's understanding of the given task \cite{brown2020languagemodelsfewshotlearners}. The presentation of high-quality examples has been shown to improve LLM performance on more complex tasks compared to simple instruction alone \cite{systematicsurveypromptengineering}. However, Few-Shot prompts are inherently challenging to implement in order to be effective. Factors such as the selection, similarity, quantity, and order of exemplars--as well as the format or placement of instructions--can substantially influence model responses \cite{promptreportsystematicsurvey}. For example, varying the order in which the task instances are demonstrated can intriguingly produce accuracy scores that vary from sub-50\% to over 90\% \cite{lu2022fantasticallyorderedpromptsthem}. Therefore, careful decisions throughout the prompt design process are critical to ensuring optimal LLM behavior.

\begin{figure}[H]
    \vskip -0.01in
    \centering
    \subfloat[Example of Zero-Shot technique.]{ % Subfigure 1
        \includegraphics[width=0.26\linewidth]{images/zero-shot.png}
        \vspace{0.3cm}
        \label{fig:zero}
    }   \hspace{0.3cm}
    \subfloat[Example of One-Shot technique.]{ % Subfigure 2
        \includegraphics[width=0.26\linewidth]{images/one-shot.png}
        \vspace{0.3cm}
        \label{fig:one}
    } \hspace{0.3cm}
    \subfloat[Example of Few-Shot technique.]{ % Subfigure 3
        \includegraphics[width=0.26\linewidth]{images/few-shot.png}
        \label{fig:few}
    }
    \vskip -0.01in
    \caption{Comparison between the Zero-Shot, One-Shot and Few-Shot prompting techniques with examples on the English-to-French translation task \cite{brown2020languagemodelsfewshotlearners}.}
    \label{fig:zero-one-few}
    \vskip -0.09in
\end{figure}

\subsubsection{Chain-of-Thought Prompting}

Despite their undeniable potential, Large Language Models often encounter difficulties when challenged with questions that are not directly answerable without intermediate inferences. The Chain-of-Thought (CoT) prompting technique was introduced in order to address this issue by encouraging the model to articulate its thought process through a sequence of immediate outputs, before generating the final answer \cite{systematicsurveypromptengineering}. Experimental results have shown that the employment of these reasoning chains improves LLM performance--often to a remarkable degree--under various non-trivial tasks, including multi-hop question-answering, arithmetic, commonsense and, symbolic reasoning problems (\cite{cotpromptingelicitsreasoning}; \cite{understandingchainofthoughtpromptingempirical}).

\begin{figure}[H]
    \vskip -0.01in
    \centering
    \includegraphics[width=0.6\linewidth]{images/cot.png}
    \caption{Application of Chain-of-Thought prompting for arithmetic reasoning \cite{cot-figure}.} \label{fig:cot}
    \vskip -0.09in
\end{figure}

Chain-of-Thought prompting can be incorporated into both Zero-Shot and One-Shot/Few-Shot scenarios. In the Zero-Shot setting, a simple instruction like ``Let's think step by step.'' is added to the prompt to encourage task decomposition (\cite{kojima2023largelanguagemodelszeroshot}; \cite{automaticchainthoughtprompting}).  In the One-Shot or Few-Shot settings, each demonstration typically consists of a question followed by a manually designed natural language rationale that leads to the final answer \cite{cotpromptingelicitsreasoning}. Automatic Chain-of-Thought (Auto-CoT) \cite{automaticchainthoughtprompting} extends this by using the LLM itself to generate reasoning chains for demonstration examples, removing the need for manual rationale construction while maintaining the performance benefits of step-by-step reasoning.

The effectiveness of CoT prompting has been shown to scale with model size---smaller models often fail to produce coherent reasoning chains, while larger models (above approximately 100B parameters) consistently benefit from the technique \cite{cotpromptingelicitsreasoning}. This observation suggests that CoT activates latent reasoning capabilities that emerge only at sufficient model scale.

\subsubsection{Self-Consistency}

Self-Consistency \cite{wang2023selfconsistency} addresses a fundamental limitation of Chain-of-Thought prompting: the stochastic nature of LLM generation means that a single reasoning chain may follow a suboptimal path. Self-Consistency samples multiple diverse reasoning chains by setting a positive temperature during generation, and then selects the final answer by majority vote over the set of generated answers:
\begin{equation}
    \hat{a} = \arg\max_{a} \sum_{i=1}^{n} \mathbf{1}[a_i = a]
\end{equation}
where $a_i$ is the answer derived from the $i$-th reasoning chain and $n$ is the total number of sampled chains. This approach is motivated by the intuition that correct answers tend to converge across multiple reasoning paths, while incorrect answers are distributed more randomly. Self-Consistency has demonstrated improvements of 1--23 percentage points across arithmetic, commonsense, and symbolic reasoning benchmarks.

In our system, the council deliberation architecture in S2 can be viewed as a structured variant of Self-Consistency, where multiple independent ``reasoning chains'' (persona-specific analyses) vote on the final classification. However, our approach differs in that each chain is generated by a specialized persona with distinct analytical biases, rather than by sampling from the same prompt at different temperatures.

\subsubsection{Tree-of-Thought}

Tree-of-Thought (ToT) prompting \cite{yao2023tree} generalizes Chain-of-Thought from a linear reasoning chain to a \textit{branching} reasoning structure, where the model explores multiple intermediate reasoning steps and evaluates each candidate step before committing to a path. Formally, ToT decomposes a problem into a sequence of ``thought'' steps, generates multiple candidate thoughts at each step, evaluates them using the LLM itself as a value function, and searches through the resulting tree using breadth-first search (BFS) or depth-first search (DFS):
\begin{enumerate}
    \item \textbf{Thought decomposition}: Break the problem into sequential intermediate steps.
    \item \textbf{Thought generation}: At each step, generate $k$ candidate thoughts (continuations).
    \item \textbf{State evaluation}: Use the LLM to assess whether each partial solution is promising, impossible, or uncertain.
    \item \textbf{Search}: Navigate the tree using BFS, DFS, or beam search to find the best complete solution path.
\end{enumerate}

ToT is particularly effective for problems that require planning, exploration, or lookahead---such as game playing, creative writing, or multi-step mathematical proofs. While our system does not implement full tree search, the Self-Refine loop in S1 (Generator $\to$ Critic $\to$ Refiner) can be viewed as a depth-limited tree search where the Critic evaluates partial solutions and the Refiner generates improved candidates based on structured feedback.

\subsubsection{Algorithm of Thoughts (AoT)}

Algorithm of Thoughts~\cite{sel2024aot} takes a fundamentally different approach from ToT by embedding algorithmic exploration strategies \textit{within} a single LLM query rather than requiring multiple external calls. AoT prompts the LLM to simulate the search process of algorithms like depth-first search (DFS) internally, maintaining an in-context ``search tree'' that it navigates step by step. Key advantages include:
\begin{itemize}
    \item \textbf{Reduced query count:} While ToT requires $O(b^d)$ LLM calls (where $b$ is the branching factor and $d$ the depth), AoT achieves comparable or superior performance with 1--2 queries.
    \item \textbf{Implicit backtracking:} The LLM learns to abandon unpromising branches and backtrack---a capability that emerges from being prompted with algorithmic examples.
    \item \textbf{Scalability:} By avoiding the combinatorial explosion of external tree search, AoT is practical for real-time applications.
\end{itemize}

\subsubsection{Graph of Thoughts (GoT)}

Graph of Thoughts~\cite{besta2024got} generalizes the reasoning structure beyond trees to arbitrary \textit{directed acyclic graphs} (DAGs). In GoT, individual LLM thoughts are vertices and dependencies between them are edges, enabling operations that are impossible in linear or tree structures:
\begin{itemize}
    \item \textbf{Aggregation:} Multiple independent thoughts can be merged into a single, refined thought (e.g., combining partial solutions from different reasoning paths).
    \item \textbf{Refinement:} Any thought can be iteratively improved based on evaluation feedback without restarting the entire reasoning chain.
    \item \textbf{Branching and looping:} Unlike trees, the graph structure allows cycles (bounded iteration) and fan-in operations (combining insights from multiple branches).
\end{itemize}
GoT achieves significant improvements over ToT on tasks requiring decomposition and recombination, such as sorting, set operations, and document merging.

\subsubsection{Reverse Exclusion Graph-of-Thought (ReX-GoT)}

ReX-GoT~\cite{zheng2024rexgot} adapts graph-based reasoning specifically for multi-choice tasks through a three-stage reverse exclusion process:
\begin{enumerate}
    \item \textbf{Option Exclusion:} Instead of directly selecting the correct answer, the LLM systematically eliminates implausible options with explicit justification for each exclusion.
    \item \textbf{Error Analysis:} The remaining candidates are subjected to adversarial error analysis, where the LLM attempts to find flaws in the reasoning for each surviving option.
    \item \textbf{Combination:} Results from exclusion and error analysis are combined in a graph structure to produce the final answer with calibrated confidence.
\end{enumerate}
This reverse reasoning strategy is particularly effective for commonsense and dialogue inference tasks, where forward reasoning often introduces confirmation bias.

\subsubsection{Multi-Agent Prompting Techniques}

Beyond single-model prompting strategies, multi-agent prompting techniques leverage interactions between multiple LLM instances to improve output quality:

\paragraph*{Debate.} Du et al.~\cite{du2023debating} propose a framework where multiple LLM instances engage in iterative debate, each generating responses and then revising their answers after reading others' arguments. Through multiple rounds of debate, models converge toward more factually accurate and logically consistent answers. The key insight is that exposure to alternative perspectives forces each model to critically examine its own reasoning, reducing hallucinations and improving factual accuracy by 20--30\% on benchmarks like TruthfulQA.

\paragraph*{Persona Fitting and Role-Play.} Assigning distinct personas or expert roles to LLM instances elicits specialized reasoning behaviors. Each persona receives a tailored system prompt that defines its expertise, evaluation criteria, and argumentative stance. Our Anti-Echo Chamber council (Section~\ref{sec:system-overview}) is a direct application of persona fitting: the Prosecutor, Defense Attorney, Literalist, and Stance Analyst each bring genuinely different analytical frameworks to the conspiracy detection task, ensuring systematic evaluation of both confirming and disconfirming evidence.

\vspace{0.5em}
\noindent\textit{In our experiments, we explored all of the prompting techniques discussed in this section during the development trajectory (Section~\ref{sec:dev-trajectory}), including CoT variants, multi-agent debate, and persona-based reasoning. Through systematic evaluation, we selected the Dynamic Discriminative Chain-of-Thought (DD-CoT) approach combined with the Anti-Echo Chamber council as the most effective combination for psycholinguistic marker extraction and conspiracy detection.}

\subsection{Prompt Formatting: XML vs. Markdown}
\label{sec:prompt-format}

Beyond the content of a prompt, its \textit{structural format} significantly impacts model performance. Different LLM providers exhibit distinct preferences for how instructions are organized:

\subsubsection{XML-Structured Prompts}

XML tags provide explicit semantic boundaries that help models parse hierarchical instructions. This format is particularly effective for Anthropic's Claude models, which are specifically trained on XML-formatted instructions:
\begin{lstlisting}[language=XML, caption=XML prompt structure (Claude-optimized)]
<system_directive>
  <role>You are a Forensic Narrative Analyst.</role>
  <extraction_ontology>
    <category name="Actor">The Agent of Change</category>
    <category name="Action">The Mechanism</category>
  </extraction_ontology>
  <segmentation_rules>
    <rule type="verbatim">STRICT VERBATIM</rule>
  </segmentation_rules>
</system_directive>
\end{lstlisting}

The advantages of XML formatting include unambiguous section delineation, nested structure for complex taxonomies, and explicit closing tags that prevent instruction bleed between sections. This design choice is motivated by three converging lines of evidence:

\paragraph{Structured Boundary Enforcement.} LLM-integrated applications are vulnerable to \textit{indirect prompt injection}, where the boundary between instructions and data is blurred \cite{greshake2023indirect}. In our pipeline, user-submitted Reddit text is injected into prompts alongside complex multi-section instructions. XML tags create unambiguous structural delimiters that prevent the model from confusing document content with system directives, a critical concern when processing adversarial or conspiratorial text.

\paragraph{Hierarchical Parsing.} Recent work formalizes XML prompting as grammar-constrained interaction, demonstrating that tree-structured prompts enable LLMs to parse complex multi-part instructions more reliably than flat text \cite{alpay2025xmlprompting, sambaraju2025xmlstructured}. Both Anthropic \cite{anthropic2024xml} and OpenAI \cite{openai2024prompting} explicitly recommend XML tags for structuring complex prompts, noting improved accuracy and reduced misinterpretation.

\subsubsection{Markdown-Structured Prompts}

Markdown formatting uses headers, bullet points, and emphasis to organize instructions. This format is preferred by OpenAI's GPT models:
\begin{lstlisting}[caption=Markdown prompt structure (GPT-optimized)]
# ROLE
You are the **PROSECUTOR**.

## PRIMARY OBJECTIVE
### MAXIMIZE RECALL

## INDICTMENT CRITERIA
### 1. INSTITUTIONAL CAPTURE RULE (CRITICAL)
If an institution is framed as:
- Working against the people
- Controlled by hidden interests
-> **CONSPIRACY**
\end{lstlisting}

Markdown is more human-readable and compact, but lacks the strict boundary enforcement of XML tags. The choice between formats represents a practical consideration in prompt engineering: maintaining dual-format prompts enables model-agnostic deployment, as demonstrated in our system's migration from Claude to GPT (Section~\ref{sec:implementation-detail}).


\section{Pipeline Components}
\label{sec:models}

\subsection{Language Model}

All LLM-mediated components (Generator, Critic, Refiner, Council Jurors, Judge) use \textbf{GPT-5.2} (OpenAI, 2025). The model is accessed via the OpenAI API with the following configuration:
\begin{itemize}
    \item \textbf{Temperature:} $\tau = 0.6$ for generative components (Generator, Refiner, Council), $\tau = 0.4$ for the Literalist juror (promoting deterministic, conservative reasoning), and $\tau = 0.0$ for the Judge (deterministic aggregation).
    \item \textbf{Max output tokens:} 16,384 for the Generator (to accommodate DD-CoT chains for long documents), 8,192 for other components.
    \item \textbf{Reasoning mode:} Disabled. All reasoning is explicitly orchestrated through DD-CoT and council deliberation rather than delegated to model-internal reasoning chains.
\end{itemize}

\subsection{Agentic Framework: LangGraph}

The agentic workflow is orchestrated using \textbf{LangGraph} \cite{langgraph2024}, which implements the pipeline as a directed graph of computational nodes. Each node encapsulates a single responsibility (generation, critique, refinement, verification, forensic profiling, voting, judging) and communicates with adjacent nodes through typed state objects. Key advantages of LangGraph include:
\begin{itemize}
    \item \textbf{Conditional routing:} Nodes that fail validation (e.g., Verifier finding unresolvable spans) trigger fallback paths rather than silent failures.
    \item \textbf{Parallel execution:} Council jurors execute concurrently, reducing latency compared to sequential deliberation.
    \item \textbf{State persistence:} The full pipeline state (including intermediate LLM outputs) is serializable for debugging and reproducibility.
\end{itemize}

\subsection{Structured Output: PydanticAI}

All LLMs interface through \textbf{PydanticAI} \cite{pydanticai2024}, a framework that wraps LLM calls with typed Pydantic schemas for both input and output. This ensures that every LLM response is parsed, validated, and type-checked before propagating to downstream nodes. PydanticAI implements automatic retry with schema error feedback: if the LLM produces output that fails Pydantic validation, the error message is appended to the prompt and the call is retried (up to 3 attempts).

\subsection{Retrieval Infrastructure}

Few-shot examples are stored and retrieved using \textbf{ChromaDB} \cite{chromadb2023}. Documents are embedded using \texttt{all-MiniLM-L6-v2} for initial retrieval, then reranked using \texttt{BAAI/bge-reranker-v2-m3} \cite{bge_m3}. The two-stage retrieve-then-rerank approach balances recall (embedding similarity over the full corpus) with precision (cross-encoder scoring over a candidate shortlist). Over-retrieval factors are set to $3\times$ for S1 and $4\times$ for S2 to ensure sufficient hard-negative candidates survive reranking.

\subsection{Experiment Tracking}

All experiments, prompt versions, and evaluation metrics are tracked using \textbf{MLflow} \cite{mlflow2024}. GEPA's population management leverages MLflow's model registry to version prompt templates and track lineage through crossover and mutation operations.

\section{Experimental Setup}
\label{sec:exp-setup}

\subsection{Training--Evaluation Split}

The SemEval-2026 Task 10 organizers released a training set containing 3,682 rehydrated Reddit documents. After preprocessing (Section~\ref{sec:dataset}), we retain 3,271 unique documents. We reserve 10\% of the training set as a held-out development split, stratified by label and subreddit, for GEPA prompt optimization and ablation studies.

\subsection{Inference Configuration}

All inference is performed with \textbf{GPT-5.2} through the OpenAI API. No model fine-tuning is performed; all improvements derive from prompt engineering and agentic workflow design. The system processes documents sequentially, with parallel execution only within the S2 council (four jurors run concurrently).

\subsection{Baselines}

We compare against two baselines:
\begin{enumerate}
    \item \textbf{Zero-shot GPT-5.2:} A single-pass prompt with no retrieval, no self-refinement, and no council deliberation. This isolates the contribution of the agentic workflow.
    \item \textbf{Retrieval-only:} Zero-shot + contrastive few-shot retrieval, without the self-refine loop (S1) or council architecture (S2). This isolates the contribution of retrieval from agent orchestration.
\end{enumerate}

\subsection{Evaluation Protocol}

For the final submission, models are evaluated on a hidden test set provided by the task organizers. The official metrics are:
\begin{itemize}
    \item \textbf{S1:} Macro F1 over the five marker types.
    \item \textbf{S2:} Macro F1 over the binary labels.
\end{itemize}

Development evaluations also report accuracy, class-level precision and recall, and false-positive analysis (Reporter Trap rate).

\subsection{Pipeline Components}
\label{sec:pipeline-components}

All final experiments use OpenAI \textbf{GPT-5.2} accessed via Pydantic-AI \cite{pydanticai2024} for schema-constrained generation. Stateful agent workflows are implemented as directed acyclic graphs using \textbf{LangGraph} \cite{langgraph2024}, where each node maintains typed state with explicit field annotations enabling deterministic transitions. The few-shot retrieval component uses \textbf{ChromaDB} \cite{chromadb2023} with OpenAI text-embedding-3-small embeddings (1536 dimensions) and \textbf{Maximal Marginal Relevance (MMR)} reranking \cite{carbonell1998mmr} using the \textbf{BAAI/bge-reranker-v2-m3} cross-encoder. MMR balances relevance against diversity via:

\begin{equation}
    \text{MMR} = \arg\max_{d_i \in R \setminus S}\big[\lambda \cdot \text{Rel}(d_i, q) - (1-\lambda) \cdot \max_{d_j \in S}\text{Sim}(d_i, d_j)\big]
\end{equation}

where $R$ is the candidate set, $S$ the already-selected documents, and $\lambda=0.7$ biases toward relevance while preventing near-duplicate few-shots. Relevance scores from the cross-encoder are min-max normalized per batch to the $[0,1]$ range. S1 retrieval over-retrieves $3\times$ candidates before reranking, while S2 uses $4\times$ to ensure higher-quality hard negatives. All LLM calls execute asynchronously with exponential backoff retry logic (base 2s, max 5 retries).

We employ differential temperature settings:
\begin{itemize}
    \item $\tau=0.7$ for the DD-CoT Generator to encourage diverse candidate exploration.
    \item $\tau=0.4$ for Council Jurors to balance creative reasoning with verdict consistency.
    \item $\tau=0.0$ for the Critic, Refiner, and Judge to enforce deterministic, reproducible auditing.
\end{itemize}

This stratification reflects each agent's functional role: generative nodes benefit from sampling diversity to avoid mode collapse over marker types, while evaluative nodes require strict adherence to textual evidence. For prompt optimization, we utilize \textbf{GEPA} \cite{agrawal2025gepa} integrated with MLflow \cite{mlflow2024}, using a passthrough injection pattern to tunnel gold labels through the prediction wrapper for custom scoring. We conduct optimization runs targeting S1 and S2 system prompts with population sizes of 20--30 candidates and 40--80 trials per run, alternating between training and development splits to ensure generalization.
\section{Development Trajectory}
\label{sec:development-trajectory}

This section documents the iterative development process spanning five months (September 2025--January 2026) through 42 git commits, revealing how the architecture evolved from naive baselines to the current agentic system.


\subsection{Evolution Timeline}

The development followed six distinct phases, each representing a paradigm shift in approach:

\paragraph{Phase 1: Data Understanding (September--October 2025).}
Initial work focused on exploratory data analysis (EDA) to understand the task characteristics. Key insights from this phase:
\begin{itemize}[noitemsep]
    \item Discovered severe class imbalance: \textit{Evidence} and \textit{Victim} markers are rare ($<$15\% of spans)
    \item Identified the \textit{Reporter Trap}: Reddit Submission Statements often describe conspiracies without endorsing them
    \item Built the data pipeline for rehydrating redacted social media posts
    \item Developed comprehensive EDA visualizations documenting marker distributions, text length correlations, and annotation overlaps (Section~\ref{sec:eda})
\end{itemize}

\paragraph{Phase 2: Baseline Establishment (October 2025).}
AWS Bedrock integration with Claude Sonnet 4.5~\cite{anthropic2025claude} established initial LLM baselines:
\begin{itemize}[noitemsep]
    \item Zero-shot prompting achieved S1 F1 = 0.12, S2 Accuracy = 0.71
    \item Prompt sweeping tested systematic variations across temperature, persona framing, and output formats
    \item XML tags improved Claude's adherence to structured output requirements
    \item Claude Haiku 4 was used for rapid prototyping due to its lower cost and faster inference
\end{itemize}

\paragraph{Phase 3: Prompt Engineering (October--November 2025).}
Intensive prompt iteration with focus on span extraction precision:
\begin{itemize}[noitemsep]
    \item Developed deterministic span verification to eliminate hallucinated extractions
    \item Introduced the ``hybrid approach'': LLM extraction + programmatic validation
    \item Achieved first significant S1 improvement (F1: 0.12 $\rightarrow$ 0.20) through granularity constraints
    \item Added few-shot examples with explicit rationales explaining \textit{why} each span was labeled
    \item Experimented with XML-structured vs. plain-text prompts, finding XML superior for Claude
\end{itemize}

\paragraph{Phase 4: Agentic Architecture (November--December 2025).}
Transition from monolithic prompts to multi-agent workflows:
\begin{itemize}[noitemsep]
    \item Implemented Pydantic-AI~\cite{pydanticai2024} for structured outputs, eliminating JSON parsing failures
    \item Added self-consistency ensemble ($k=3$) with majority voting
    \item Developed LangGraph~\cite{langgraph2024} ``Legislator-Judge'' system for S2 with adversarial debate
    \item Introduced ReX-GoT (Reasoning \& Execution Graph-of-Thought) for complex reasoning chains
    \item Built RAG system with batch inference for few-shot retrieval using ChromaDB~\cite{chromadb2023}
\end{itemize}

\paragraph{Phase 5: Automated Optimization (December 2025--January 2026).}
GEPA integration for systematic prompt improvement:
\begin{itemize}[noitemsep]
    \item Integrated MLflow for experiment tracking and prompt versioning
    \item Developed custom scorers with ``Trojan Horse'' pattern for rich feedback (Section~\ref{sec:gepa-details})
    \item Dynamic RAG context injection based on query similarity
    \item Phased optimization: Generator $\rightarrow$ Critic $\rightarrow$ Refiner (S1), Council $\rightarrow$ Judge (S2)
\end{itemize}

\paragraph{Phase 6: Architecture Consolidation (January 2026).}
Migration to OpenAI GPT-5.2 and final system refinement:
\begin{itemize}[noitemsep]
    \item Ported prompts from XML (Claude-optimized) to Markdown (GPT-optimized)
    \item Simplified S1 to single DD-CoT agent + deterministic tools (reduced token cost by 60\%)
    \item Added appeal/retrial mechanism in S2 Judge for borderline cases
    \item Finalized Anti-Echo Chamber parallel council architecture
\end{itemize}


\subsection{Performance Progression}

Table~\ref{tab:dev-performance} summarizes key milestones on the held-out development set, mapped to the git commit timeline.

\begin{table}[H]
\centering
\caption{Key performance milestones mapped to development timeline. Best results in bold.}
\label{tab:dev-performance}
\begin{tabular}{@{}lllcc@{}}
\toprule
\textbf{Date} & \textbf{Commit} & \textbf{Innovation} & \textbf{S1 F1} & \textbf{S2 Acc} \\
\midrule
Oct 12 & Bedrock Integration & Zero-Shot Baseline & 0.12 & 0.71 \\
Oct 23 & Prompt Sweep & Few-Shot + CoT & 0.14 & 0.67 \\
Oct 28 & Deterministic Verifier & Hybrid Extraction & 0.18 & 0.74 \\
Oct 30 & Granularity Fix & \textbf{S1 Highscore v1} & 0.20 & 0.75 \\
Nov 12 & Pydantic-AI & Structured Outputs & 0.20 & 0.79 \\
Nov 28 & Self-Consistency & Ensemble ($k=3$) & 0.22 & 0.78 \\
Nov 29 & LangGraph S2 & Legislator-Judge & 0.22 & 0.78 \\
Dec 03 & RAG System & Contrastive Few-Shot & \textbf{0.24} & 0.79 \\
Dec 29 & GEPA Alpha & Automated Optimization & 0.23 & \textbf{0.79} \\
Jan 26 & DD-CoT System & Final Architecture & \textbf{0.24} & 0.80 \\
Jan 31 & Simplified S1 & Single Agent + Tools & 0.24 & \textbf{0.81} \\
\bottomrule
\end{tabular}
\end{table}

\begin{figure}[H]
\centering
\begin{tikzpicture}[
    phase/.style={draw, rounded corners, fill=blue!8, minimum width=1.6cm, minimum height=1.2cm, align=center, font=\footnotesize},
    final/.style={draw, rounded corners, fill=green!15, minimum width=2cm, minimum height=1.2cm, align=center, font=\footnotesize},
    arr/.style={-{Stealth[length=3mm]}, thick, gray},
    date/.style={font=\scriptsize\itshape, text=gray}
]
    \node[phase] (p1) at (0,0) {EDA\\Analysis};
    \node[phase] (p2) at (2.5,0) {Bedrock\\LLM};
    \node[phase] (p3) at (5,0) {Prompt\\Eng.};
    \node[phase] (p4) at (7.5,0) {Pydantic\\AI};
    \node[phase] (p5) at (10,0) {LangGraph\\Agents};
    \node[phase] (p6) at (12.5,0) {GEPA\\+ OpenAI};
    
    \node[final] (pf) at (12.5,-2) {DD-CoT +\\Anti-Echo\\Chamber};
    
    \draw[arr] (p1) -- (p2);
    \draw[arr] (p2) -- (p3);
    \draw[arr] (p3) -- (p4);
    \draw[arr] (p4) -- (p5);
    \draw[arr] (p5) -- (p6);
    \draw[arr] (p6) -- (pf);
    
    \node[date] at (0,-0.95) {Oct 5};
    \node[date] at (2.5,-0.95) {Oct 12};
    \node[date] at (5,-0.95) {Oct 28};
    \node[date] at (7.5,-0.95) {Nov 12};
    \node[date] at (10,-0.95) {Nov 29};
    \node[date] at (12.5,-0.95) {Jan 14};

    \draw[thick, gray!50, dashed] (-1.2,1) -- (13.7,1);
    \node[font=\scriptsize, text=gray] at (-0.5,1.25) {Sep 2025};
    \node[font=\scriptsize, text=gray] at (13,1.25) {Feb 2026};
\end{tikzpicture}
\caption{System evolution timeline derived from 42 git commits across 5 months.}
\label{fig:dev-timeline}
\end{figure}


\subsection{Error Analysis}
\label{sec:error-analysis}

Systematic error analysis on the development set revealed four critical failure modes that informed our architectural decisions.

\paragraph{S1: Label Confusion.} Exploratory data analysis revealed that \textbf{Action $\leftrightarrow$ Effect} is the most confusing pair, followed by \textbf{Actor $\leftrightarrow$ Victim} in passive constructions. For example, in ``The people are being poisoned by the government,'' the LLM frequently classified ``the government'' as both Actor (correct) and as a contextual entity without a label. Similarly, ``being poisoned'' straddles the boundary between Action (the mechanism) and Effect (the outcome). The DD-CoT discriminative reasoning directly addresses this by forcing explicit ``why NOT other label'' justification.

\paragraph{S1: Phantom Spans.} Early LLM approaches hallucinated spans not present in the source text. For instance, the model might extract ``government conspiracy'' when the source text only contained ``the government did X'' without the word ``conspiracy.'' The deterministic Verifier cascade (exact match $\rightarrow$ case-insensitive $\rightarrow$ normalized $\rightarrow$ fuzzy $\rightarrow$ LCS), introduced in the October 28 commit, reduced phantom spans to near-zero while preserving recall.

\paragraph{S2: The Reporter Trap.} Single-agent classifiers confused \textit{topic presence} with \textit{endorsement}. News articles and Reddit Submission Statements discussing conspiracy theories were systematically misclassified as ``conspiracy'' even when the author was merely reporting on conspiratorial claims. The Anti-Echo Chamber architecture, with its dedicated Defense Attorney and Reporter Defense prompting, specifically addresses this failure mode.

\paragraph{S2: Sequential Debate Bias.} The initial LangGraph ``Legislator-Judge'' system (November 29) used sequential debate where later agents saw earlier arguments, introducing ordering bias---later agents disproportionately agreed with earlier agents. The parallel council architecture (January 2026) ensures independent voting by executing all four personas concurrently via \texttt{asyncio.gather} without information leakage.


\subsection{Key Architectural Insights}

The iterative development process yielded five key insights that shaped the final architecture:

\begin{enumerate}
    \item \textbf{Single Well-Prompted Agent $>$ Ensemble for S1:} Self-consistency with $k=3$ provided diminishing returns compared to the DD-CoT + Critic + Refiner pipeline, which achieves comparable quality with fewer tokens by decomposing the problem into specialized reasoning stages.
    
    \item \textbf{Multi-Agent Essential for S2:} Unlike S1, S2 benefits from genuine perspective diversity. The parallel council prevents the ``echo chamber'' effect where a single agent locks into an initial interpretation, particularly for borderline cases involving reporting vs. endorsement.
    
    \item \textbf{Structured Outputs Critical:} The transition to Pydantic-AI (Phase 4) yielded immediate gains (S1: 0.14 $\rightarrow$ 0.20, S2: 0.77 $\rightarrow$ 0.79) by eliminating JSON parsing failures and enforcing schema constraints.
    
    \item \textbf{RAG Requires Contrastive Examples:} Standard few-shot retrieval based on similarity alone was insufficient. Hard negative mining and contrastive rationale generation (Phase 5) taught discrimination rather than pattern matching.
    
    \item \textbf{Confidence Calibration Matters:} Programmatic confidence damping based on council consensus enabled reliable uncertainty quantification for downstream decision-making, preventing overconfident misclassifications.
\end{enumerate}
