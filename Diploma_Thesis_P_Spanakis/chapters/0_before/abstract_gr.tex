\chapter*{Περίληψη}

Η παρούσα διπλωματική εργασία παρουσιάζει ένα πρακτορικό σύστημα βασισμένο σε Μεγάλα Γλωσσικά Μοντέλα (ΜΓΜ) για το SemEval-2026 Task 10, το οποίο συνδυαστικά (i) εξάγει ψυχογλωσσικούς δείκτες συνωμοσίας ως τμήματα κειμένου και (ii) ταξινομεί εάν ένα σχόλιο στο Reddit υιοθετεί μια θεωρία συνωμοσίας. Ο σχεδιασμός διαχωρίζει τη σημασιολογική εξαγωγή από τον ντετερμινιστικό εντοπισμό εύρους και χρησιμοποιεί ένα συμβούλιο Αντι-Ηχοθαλάμου (Anti-Echo Chamber) για τη μείωση των ψευδώς θετικών αποτελεσμάτων σε περιεχόμενο αναφοράς και αντίκρουσης. Για το Υποέργο 1 (Εξαγωγή Δεικτών), εισάγουμε τη Δυναμική Διακριτική Αλυσίδα Σκέψης (DD-CoT) εντός ενός βρόχου Αυτο-Βελτίωσης (Self-Refine), όπου ένας Παραγωγός (Generator) προτείνει ετικετοποιημένες συμβολοσειρές δεικτών, ένας Ενισχυμένος Κριτικός (Enhanced Critic) τις ελέγχει, ένας Βελτιωτής (Refiner) εφαρμόζει διορθώσεις και ένας Ντετερμινιστικός Επαληθευτής (Deterministic Verifier) αγκυρώνει κάθε συμβολοσειρά σε ακριβείς θέσεις χαρακτήρων. Για το Υποέργο 2 (Ανίχνευση Συνωμοσίας), δομούμε την ταξινόμηση ως αντιδικιστική διαβούλευση μέσω ενός Παράλληλου Συμβουλίου τεσσάρων εξειδικευμένων προσώπων (Εισαγγελέας, Συνήγορος Υπεράσπισης, Κυριολέκτης και Αναλυτής Στάσης), των οποίων οι ανεξάρτητες ψήφοι συγκεντρώνονται από έναν Βαθμονομημένο Δικαστή με συντηρητικούς κανόνες εμπιστοσύνης. Και τα δύο υποέργα αξιοποιούν Αντιπαραβολική Ανάκτηση Παραδειγμάτων (Contrastive Few-shot Retrieval), συμπεριλαμβανομένης στρωματοποιημένης δειγματοληψίας για την εξαγωγή δεικτών και εξόρυξης δύσκολων αρνητικών παραδειγμάτων για τη διάκριση στάσης. Τα πρότυπα προτροπών βελτιστοποιούνται μέσω του Γενετικού Εξελικτικού Αλγορίθμου Προτροπών (GEPA). Η πρακτορική αγωγός διπλασιάζει το μακρο-F1 του Υ1 (από 0.12 σε 0.24) και βελτιώνει το μακρο-F1 του Υ2 κατά 49\% (από 0.53 σε 0.79) σε σχέση με τη βασική γραμμή μηδενικής προτροπής του GPT-5.2, αποδεικνύοντας ότι η δομή ροής εργασίας μπορεί να υποκαταστήσει αλλαγές στο μοντέλο ή στα δεδομένα σε ψυχογλωσσικά πλαίσια ΕΦΓ. Εξ όσων γνωρίζουμε, αυτό αποτελεί την πρώτη πρακτορική μέθοδο βασισμένη σε ΜΓΜ για εξαγωγή και ανίχνευση ψυχογλωσσικών δεικτών συνωμοσίας.

\paragraph*{Λέξεις-κλειδιά ---}
Μεγάλα Γλωσσικά Μοντέλα (ΜΓΜ), πρακτορικές ροές εργασίας, ανίχνευση συνωμοσιών, ψυχογλωσσικοί δείκτες, εξαγωγή εύρους κειμένου, πολυπρακτορική διαβούλευση, αλυσίδα σκέψης, αυτο-βελτίωση, αντιπαραβολική ανάκτηση, βελτιστοποίηση προτροπών, SemEval.
